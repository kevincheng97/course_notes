\documentclass[11pt]{article}
\usepackage{notes}

\newcommand{\thiscoursecode}{PMATH 348}
\newcommand{\thiscoursename}{Complex Analysis}
\newcommand{\thisprof}{Prof. Akshaa Vatwani}
\newcommand{\me}{Kevin Cheng}
\newcommand{\thisterm}{Winter 2018}

\hypersetup
{pdfauthor={\me},
pdftitle={\thiscoursecode \thisterm Lecutre Notes},
pdfsubject={\thiscoursename}
pdflang={English}}

\begin{document}
\begin{titlepage}
\begin{centering}
{\scshape\LARGE University of Waterloo \par}
\globe
{\huge\bf \thiscoursecode}\\
{\scshape\Large \thiscoursename}\\
\vspace{.3cm}
{\scshape \thisprof~\textbullet~\thisterm \par}
\end{centering}
\sectionline
\tableofcontents
\sectionline
\thispagestyle{empty}
\end{titlepage}

\section{Complex Numbers}
\begin{definition}
A \underline{complex number} is a vector in $\R^2$. The \underline{complex plane} denoted by $\C$ is the
set of complex numbers. 
\begin{equation*}
\C = \R^2 = \bigg\{\begin{pmatrix}x\\y\end{pmatrix}: x, y \in \R \bigg\}
\end{equation*}
In $\C$, we usually write,
\begin{align*} 
0 &= \begin{pmatrix}0\\0\end{pmatrix}\\
1 &= \begin{pmatrix}1\\0\end{pmatrix}\\
i &= \begin{pmatrix}0\\1\end{pmatrix}\\
x &= \begin{pmatrix}x\\0\end{pmatrix}\\
iy &= \begin{pmatrix}0\\y\end{pmatrix}
\end{align*}
with $x, y \in \R$. If $z = x + iy, x, y \in \R$, then $x$ is called the real
part of $z$ and $y$ the imaginary part of $z$ and write
\begin{equation*}
\Re(z) = x \qquad \Im(z) = y
\end{equation*}
\end{definition}
\begin{definition}
We define the \underline{sum of two complex numbers} to be the vector sum.
\begin{align*}
(a+ib)+(c + id) &=
\begin{pmatrix}a \\ b\end{pmatrix} + 
\begin{pmatrix}c \\ d\end{pmatrix}\\
&= 
\begin{pmatrix}a + c \\ b + d\end{pmatrix} 
\end{align*}
We define the \underline{product of two complex numbers} by setting $i^2 = -1$ and by
requiring the product to be commutative, associative and distributive over the
sum.
So,
\begin{align*}
(a + bi)(c + di) &=
ac + iad + ibc + i^2bd\\
&= (ac - bd) + (ad + bc)
\end{align*}
\end{definition}
\begin{prop}[Mulitplicative Inverses]
Every complex number has a unique multiplicative inverse denoted by
$z^{-1}$.
\end{prop}
\begin{proof}
Let $z = a + i, a,b \in \R$ with $a^2 + b^2 = 0$. We want to solve for $x$ and
$y$ such that $(a + ib)(x + iy) = 1$. In other words,
\begin{align*}
&\> (ax - by) + i(ay + bx) = 1\\
\Rightarrow &\>
\begin{pmatrix}
ax - by \\ bx + ay
\end{pmatrix}
= (1,0)\\
\Rightarrow &\>
\begin{pmatrix}
a & -b\\
b & a
\end{pmatrix}
\begin{pmatrix}
x \\ y
\end{pmatrix}
= (1,0)\\
\Rightarrow &\>
\begin{pmatrix}
x \\ y
\end{pmatrix}
=
\begin{pmatrix}
a & -b\\
b & a
\end{pmatrix}^{-1}
\begin{pmatrix}
1 \\ 0
\end{pmatrix}\\
\Rightarrow &\>
\begin{pmatrix}
x \\ y
\end{pmatrix}
=
\frac{1}{a^2 + b^2}
\begin{pmatrix}
a & b\\
-b & a
\end{pmatrix}
\begin{pmatrix}
1 \\ 0
\end{pmatrix}\\
\Rightarrow &\>
\begin{pmatrix}
x \\ y
\end{pmatrix}
=
\begin{pmatrix}
\frac{a}{a^2 + b^2} \\ \frac{b}{a^2 + b^2}
\end{pmatrix}
\end{align*}
This is unique as the inverse matrix is unique.
\end{proof}
\begin{remark}
The set of complex numbers is a \underline{field} under the operations of
addition and multiplication as operations are associative, commutative
and distributive and every element has a unique inverse as before.
\end{remark}
\begin{definition}
If $z = x + iy, x, y, \in \R$, then the \underline{conjugate of $z$} is
$\bar{z} = x - iy$.
\end{definition}
\begin{definition}
We define the \underline{modulus} (or length or magnitude) of $z = x + iy, x, y \in \R$ to
be
\begin{equation*}
|z| = \sqrt{x^2 + y^2} \in \R
\end{equation*}
\end{definition}
\begin{remark}
For any $z, w \in \C$,
\begin{align*}
\bar{\bar{z}} &= z\\
z + \bar{z} &= 2\Re(z)\\
z - \bar{z} &= 2\Im(z)\\
z\cdot\bar{z} &= |z|^2\\
|z| &= |\bar{z}|\\
\bar{z+w} &= \bar z + \bar w\\
\bar{zw} &= \bar z\cdot\bar w\\
|zw| &= |z||w|
\end{align*}
\end{remark}
\begin{prop}
The following inequalities hold for any $z \in \C$.
\begin{enumerate}
\item $|\Re(z)| \leq |z|$
\item $|\Im(z)| \leq |z|$
\item $|z + w| \leq |z| + |w|$
\item $|z + w| \geq \bigg| |z| - |w|\bigg|$
\end{enumerate}
\begin{proof}
(1) and (2) follows as
\begin{equation*}
|z|^2 = \Re(z)^2 + \Im(z)^2. 
\end{equation*}
(3). Notice,
\begin{align*}
|x + iy|^2 &= (x + iy)\bar{(x + iy)}\\
&= (x + iy)(\bar{x} + \bar{iy})\\
&= x\bar x + y \bar y + x \bar y + y \bar x\\
&= |x|^2 + |y|^2 + x \bar y + y \bar x\\
&= |x|^2 + |y|^2 + 2\Re(x\bar y)\\
&\leq |x|^2 + |y|^2 + 2|x\bar y|\\
&= |x|^2 + |y|^2 + 2|x| \cdot |\bar y|\\
&= |x + y|^2
\end{align*}
Taking the square root of both sides gives the result.\\

(4). From (3), we have that
\begin{align*}
|z| = |z - w + w| &\leq |z - w| + |w|\\ 
|w| = |w - z + z| &\leq |w - z| + |z|
\end{align*}
Then, isolating $|z - w|$ implies the result. More specifically since we have the simulateous inequality,
\begin{equation*}
\begin{cases}
|z| - |w| \leq |z - w| \\
|w| - |z| \leq |z - w|
\end{cases}
\Rightarrow
|z - w| \geq \bigg| |z| - |w| \bigg|
\end{equation*}
as desired.
\end{proof}
\end{prop}
\begin{prop}
Every non-zero complex number has exactly 2 square roots.
\end{prop}
\begin{proof}
Let $z = x + iy \in \C$ with $x^2 + y^2 \neq 0, x, y, \in \R$. We want to solve
$w^2 = z$ for $w \in \C$. Say $w$ takes the form $w = u + iv, u, v \in \R$. Then
\begin{align*}
& \> w^2 = z\\
\Rightarrow & \> (u+iv)^2 = x + iy\\
\Rightarrow & \> (u^2 - v^2) + i2uv = x + iy
\end{align*}
So we have that $x = u^2 - v^2$ and $y = 2uv^2$. We can solve for $u$ and $v$.
Take the square of both sides of the second equation to get $4u^2v^2 = y^2$.
Now, we multiply the first equation by $4u^2$ to get
\begin{align*}
& \> 4u^4 - 4u^2v^2 = 4xu^2\\
\Rightarrow & \> 4u^4 - 4xu^2 - y^2 = 0
\end{align*}
This is a quadratic equation over $u^2$ so,
\begin{equation*}
u^2 = \frac{4x \pm \sqrt{16x^2 + 16y^2}}{8} = \frac{x \pm \sqrt{x^2 + y^2}}{2}
\end{equation*}
Suppose that $y \neq 0$. Then we must take the positive solution above to get
\begin{equation*}
u^2 = \frac{x + \sqrt{x^2 + y^2}}{2}
\end{equation*}
Under the assumption that $x^2 + y^2 > 0$, this solution exists. Notice we
cannot take the negative solution as it yields a negative $u^2$ which is
impossible. We can use a similar procedure to find that
\begin{equation*}
v^2 = \frac{-x + \sqrt{x^2 + y^2}}{2}
\end{equation*}
Rooting $u$ and $v$ gives 2 solutions for each. However, if $y$ is positive,
since $2uv = y$, $u$ and $v$ must take the same sign. Similarly, if $y$ is
negative, they must take different signs. In each of these cases, there are 2
solutions for $u$ and $v$. So,
\begin{equation*}
w = 
\begin{cases}
\pm \Bigg[ \bigg(\sqrt{\frac{x + \sqrt{x^2 + y^2}}{2}}\bigg) +
i\bigg(\sqrt{\frac{-x + \sqrt{x^2 + y^2}}{2}}\bigg)\Bigg] &, \> y > 0\\
\pm \Bigg[ \bigg(\sqrt{\frac{x + \sqrt{x^2 + y^2}}{2}}\bigg) -
i\bigg(\sqrt{\frac{-x + \sqrt{x^2 + y^2}}{2}}\bigg)\Bigg] &, \> y < 0\\
\pm \sqrt x &, \> x > 0,\> y = 0\\
\pm i\sqrt -x &, \> x < 0,\> y = 0
\end{cases}
\end{equation*}
\end{proof}
\begin{remark}
Let $z \in \C$. The notation $\sqrt z$ may represent either one of the square
roots of $z$ or both of the square roots.
\end{remark}
\begin{remark}
The square root doesn't distribute. Consider $z = w = -1 \in \C$.
$\sqrt{zw} \neq \sqrt z \sqrt w$.
\end{remark}
\begin{remark}
The Quadratic Formula holds true for complex polynomials. In other words, if $a,
b, c \in \C, a \neq 0$,
\begin{equation*}
az^2 + bz + c = 0 \Rightarrow z = \frac{-b \pm \sqrt{b^2 - 4ac}}{2a}
\end{equation*}
\end{remark}
\begin{definition}
If $z \in \C \setminus \{0\}$, we define the \underline{angle} (or
\underline{argument}) of $z$ to be the angle $\theta(z)$ from the positive
$x$-axis counterclockwise to $z$. In other words, $\theta(z)$ is the angle such
that
\begin{equation*}
z = |z|\big(\cos\theta(z) + i\sin\theta(z)\big).
\end{equation*}
\end{definition}
\begin{remark}
For $\theta \in \R$ (or for $\theta \in \R/2\pi$), we have that
\begin{equation*}
e^{i\theta} = \cos(\theta) + i\sin(\theta)
\end{equation*}
\end{remark}
\begin{remark}
If $z \neq 0$, we have $x = \Re(z)$, $y = \Im(z)$, $r = |z|$ and
\begin{align*}
x &= r\cos\theta\\
y &= r\sin\theta\\
\tan\theta &= \frac{y}{x},\>\text{if } x \neq 0\\
z &= re{i\theta}\\
\bar z &= re^{-i\theta}\\
z^{-1} &= \frac{1}{r}e^{-i\theta} 
\end{align*}
\end{remark}
\begin{remark}
We now have 2 representations of a complex number $z\in\C$. We say that
$z = x + iy$ is the \underline{cartesian coordinates} of $z$ and $z =
re^{i\theta}$, where $r = |z|$, is the \underline{polar form} of $z$.
\end{remark}
Consider $z = re^{i\alpha}$ and $w = se^{i\beta}$. We have,
\begin{align*}
zw &= rs(\cos\alpha + i\sin\alpha)(\sin\beta + i\cos\beta)\\
&= rs\big((\cos\alpha\cos\beta-\sin\alpha \sin\beta) +
i(\sin\alpha\cos\beta+\cos\alpha\sin\beta)\big)\\
&= rs\big(\cos(\alpha + \beta) + i\sin(\alpha + \beta)\big)\\
&= e^{i(\alpha + \beta)}
\end{align*}
which defines a formula for multiplication in polar coordinates. Notice that the
following identity known as De Moivre's Law follows.
\begin{equation*}
(re^{i\theta})^n = r^ne^{in\theta}
\end{equation*}
for all $r,\theta \in \R$, $n \in \Z$. We can use this identity to find the
$n^\text{th}$ roots of $z$. In other words, we solve $w^n = z$. We have,
\begin{align*}
&\> w^n = z\\
\Rightarrow & \> (se^{i\alpha})^n = re^{i\theta}\\
\Rightarrow & \> s^ne^{in\alpha} = re^{i\theta}
\end{align*}
so $s^n = r$ and $n\alpha = \theta + 2\pi k$ for $k \in \Z$. In other words, we
have
\begin{equation*}
(re^{i\theta}) = \sqrt[n]r e^{i(\theta + 2\pi k )/n}, \quad k = 0,\dots,n-1
\end{equation*}
\begin{remark}
When working with complex numbers, for $0 \neq z \in \C$, and for $0 < n \in
\Z$, $\sqrt[n]z$ or $z^{1/n}$ denotes either one of the $n$ roots, or the set of
all $n^{\text{th}}$ roots.
\end{remark}
\end{document}
