\documentclass[11pt]{article}
\usepackage{notes}
\usepackage{caption}

\newcommand{\thiscoursecode}{MATH 249}
\newcommand{\thiscoursename}{Introduction toi Combinatorics (Advanced)}
\newcommand{\thisprof}{David Wagner}
\newcommand{\me}{Kevin Cheng}
\newcommand{\thisterm}{Fall 2018}

\hypersetup
{pdfauthor={\me},
pdftitle={\thiscoursecode \thisterm Lecutre Notes},
pdfsubject={\thiscoursename}
pdflang={English}}

\begin{document}
\begin{titlepage}
    \begin{centering}
        {\scshape\LARGE University of Waterloo \par}
        \globe
        {\huge\bf \thiscoursecode}\\
        {\scshape\Large \thiscoursename}\\
        \vspace{.3cm}
        {\scshape Prof. \thisprof~\textbullet~\thisterm\par}
    \end{centering}
\sectionline
\tableofcontents
\sectionline
\thispagestyle{empty}

\pagebreak
\section{Introduction}
\subsection{Course Outline}
MATH 249 has two parts: first part will be {\bf Enumeration} and second part
will be {\bf Graph Theory}. We will cover everything in MATH 239, and plenty of
extra
topics.

{\bf Topics in Enumeration:} Rational power series, recurrence relations,
partial fractions. Binomial Theorem, Vandermonde convolution, multisets,
Binomial Series.Generating functions, Sum and Product lemmas, computing
averages. Compositions and rational binary languages. Rational languages
andblock patterns. The transfer matrix method, domino tilings, etc. Lattice
paths, Catalan numbers.

{\bf Topics in Graph Theory:} Basic terminology, isomorphism, examples. Paths
and cycles, connectedness, spanning trees. Planar graphs, Kuratowski’s Theorem,
graphs on other surfaces. Graph colouring, the Five-Colour Theorem.
Bipartite matching, hints at the non-bipartite case. The Matrix-Tree Theorem

\subsection{Contact Information}
David Wagner, MC 5028, dgwagner@uwaterloo.ca

\subsection{Office Hours}
To be determined

\subsection{Grading Scheme}
\begin{itemize}
	\item Homework - 15\% (Due every {\it second} Friday, starting September
		$21^{\text{th}}$, at the start of class)
	\item Midterm Exam - 30\%
	\item Final Exam - 55\%
\end{itemize}

\pagebreak
\section{Enumeration}
\subsection{Generating Functions}
Let's start with the definition of the well know Fibonacci Sequence. With $f_0 =
f_1 = 1$, we define $f_n \coloneqq f_{n-1} + f_{n-2}$ for $n \geq 2$. We
arrive at the following sequence:
\begin{center}
\captionof{table}{First 9 values of the Fibonacci Sequence}
\begin{tabular}{|l|*{50}{|l}}
	\hline
	$n$   & 0 & 1 & 2 & 3 & 4 & 5 & 6  & 7  & 8  & 9  & $\dots$\\
	$f_n$ & 1 & 1 & 2 & 3 & 5 & 8 & 13 & 21 & 34 & 55 & $\dots$\\
	\hline
\end{tabular}
\end{center}

What is $f_n$ as a function of $n$?

\begin{definition}
	A \underline{generating function} $f(x)$ is a formal power series
	\begin{equation*}
		f(x) = \sum^\infty_{n=0} a_nx^n 
	\end{equation*}
	whose coefficients give the sequence $\{a_0, a_1, \dots\}$.
\end{definition}

\begin{example}
    Consider the \underline{Geometric Series}, defined by
    \begin{equation*}
        G = 1 + z + z^2 + z^3 + \dots
    \end{equation*}
    and,
    \begin{equation*}
        zG = z + z^2 + z^3 + \dots
    \end{equation*}
    so, $G-zG = 1$, thus the generating function for $G$ is
    \begin{equation*}
        G = \frac{1}{1-z}.
    \end{equation*}
\end{example}

Now, if we consider the generating function for the Fibonacci Sequence, we have,
\begin{align*}
    F(x) &= f_0 + f_1x + \sum^\infty_{n=2} f_nx^n\\
         &= 1 + x + \sum^\infty_{n=2} (f_{n-1} + f_{n-2}) x^n\\
         &= 1 + x + x\sum^\infty_{j=1} f_jx^j + x^2\sum^\infty_{n=0} f_kx^k\\
         &= 1 + x + x\big(F(x) - 1\big) + x^2F(x)
\end{align*}
So,
\begin{equation*}
    F(x) - xF(x) - x^2F(x) = 1
\end{equation*}
or,
\begin{equation*}
    F(x) = \frac{1}{1-x-x^2}
\end{equation*}
We continue by applying partial fractions to $F(x)$. First, we look for the
roots of the denominator. Notice, we can factor our denominator as
\begin{equation*}
	1 - x - x^2 = (1 - \alpha x) (1 - \beta x)
\end{equation*}
to arrive at the auxiliary function
\begin{equation*}
	t^2 -  t - 1 = (t - \alpha) (t - \beta).
\end{equation*}
The quadratic formula gives the solution for $\alpha, \beta$:
\begin{equation*}
	\alpha, \beta = \frac{1 \pm \sqrt{1 - 4 \cdot 1 \cdot -1}}{2 \cdot 1}
	= \frac{1 \pm \sqrt{5}}{2}
\end{equation*}
Now, there exist $A,B \in \C$ such that
\begin{equation*}
	F(x) = \frac{1}{1-x-x^2} = \frac{1}{(1-\alpha x)(1-\beta x)} =
	\frac{A}{1-\alpha x} + \frac{B}{1-\beta x}.
\end{equation*}
We have,
\begin{equation*}
	1 = A(1-\beta x) + B(1-\alpha x) = (A+B) + (-A\beta - B\alpha) x
\end{equation*}
and the system,
\begin{equation*}
\begin{cases}
	A + B = 1,\\
	(-A\beta - B\alpha) = 0
\end{cases}
\Rightarrow
\begin{cases}
	A\alpha + B\alpha = \alpha,\\
	A(\beta - \alpha) = 0
\end{cases}
\Rightarrow
\begin{cases}
	A\beta + B\beta = \beta,\\
	B(\alpha - \beta) = \beta
\end{cases}
\end{equation*}
so we have 
\begin{equation*}
	A = \frac{\alpha}{\alpha-\beta} = \frac{1}{\sqrt 5} \cdot
	\frac{1+\sqrt{5}}{2} = \frac{5+\sqrt 5}{10}
\end{equation*}
and 
\begin{equation*}
	B = \frac{\beta}{\beta - \alpha} = \frac{1}{-\sqrt 5} \cdot \frac{1-\sqrt
	5}{2} = \frac{5 - \sqrt 5}{10}.
\end{equation*}
Finally, we arrive at the original generating function,
\begin{align*}
	F(x) &= \sum^\infty_{n=0} f_nx^n = \frac{1}{1-x-x^2} = \frac{A}{1-\alpha x}
	+ \frac{B}{1-\beta x}\\
	&= A \sum^\infty_{n=0} (\alpha x)^n + B\sum^\infty_{n=0} (\beta x)^n
	\tag{by geometric series}\\
	&= \sum^\infty_{n=0} (A\alpha^n + B\beta^n)x^n.
\end{align*}
so, $f_n = A\alpha^n + B\beta^n$ for all $n\in\N$.

\subsection{Partial Lists}
\begin{theorem} \label{theorem2.1.2}
	The number of possibilities to arrange $n$ (distinguishable) objects in a
	row (called \underline{permutations}) is
	\begin{equation*}
		n! = n \cdot (n-1) \cdot \dots \cdot 2 \cdot 1.
	\end{equation*}
\end{theorem}
\begin{proof}
	There are obviously $n$ choices for the first position, then $n-1$ remaining
	choices for the second position, $n-2$ for the third position, etc.
	Continuing on, one obtains the stated formula.
\end{proof}
\begin{definition}
	Let $S$ be a set of size $|S| = n$. A \underline{partial set of $S$} of
	length $k$ is a sequence $s_1, s_2, \dots, s_k$ with each $s_i, s_j \in S$
	and $i \neq j$ having $s_i \neq s_j$.
\end{definition}
\begin{theorem} \label{theorem2.1.3}
	The number of partial lists of length $k$ from a set of size $n$ is
	\begin{equation*}
		\frac{n!}{(n-k)!} = n(n-1)(n-2)\dots (n-k+1).
	\end{equation*}
\end{theorem}
\begin{proof}
	The proof is essentially the same as for \cref{theorem2.1.2}: for the first
	element, there are $n$ possible choices, then $n-1$ for the second
	element, etc. For the last element, there are $n-k+1$ choices left.
\end{proof}
\begin{definition}
	A \underline{$k$-subset of a set $S$} where $|S| = n$ is a
	collection of elements $\{s_1, \dots, s_k\}$ and no two are equal, for each
	$s_1 \in S$.
\end{definition}
\begin{theorem} \label{theorem2.1.4}
The number of $k$-subsets of a set $S$ where $|S| = n$ is
\begin{equation*}
	\binom{n}{k} = \frac{n!}{k!(n-k!)}
\end{equation*}
\end{theorem}
\begin{proof}
	Let $C(n,k)$ be the number of possibilities that we are looking for. Once
	the $k$ elements have been chosen (for which there are $C(n,k)$ possible
	ways), we has $k!$ possibilities (by \cref{theorem2.1.2}) to arrange them in
	a sequence. Therefore, $C(n,k) \cdot k!$ is exactly the number of possible
	sequences of $k$ distinct elements, for which we have the formula
	\begin{equation*}
		C(n,k) \cdot k! = \frac{n!}{(n-k)!} \tag{by \cref{theorem2.1.3}}
	\end{equation*}
	and the result follows immediately.
\end{proof}
\begin{definition}
	The \underline{multiplicity of an element $s$ in set $S$} is the number of
	times $s$ is in the set $S$, denoted by $\mu(s)$.
\end{definition}
\begin{definition}
	A \underline{multiset of set $S$} is a pair $(S, \mu)$ where
	$\mu:S\to\N_{>0}$ is a mapping of the multiplicity of each element in $S$.
\end{definition}
\begin{theorem} \label{theorem2.1.5}
	The number of multisets of size $n$ with $t$ elements is 
	\begin{equation*}
		\binom{n+t-1}{t-1} = \binom{n+t-1}{n}.
	\end{equation*}
\end{theorem}
\begin{proof}
	This popular proof takes advantage of an illustration with stars and bars.
	Suppose we represent the multiplicity of each element (up to ordering) with
	stars, separating each element with stars. Note that since multiplicity is
	non-negative, we can place bars consecutively. For example, when $n = 7$ and
	$k = 5$, the tuple $(4,0,1,2,0)$ representing multiplicity can be illustrated
	by the following diagram:
	\begin{equation*}
		\star \star \star \star | | \star | \star \star \> |
	\end{equation*}
	This establishes a one-to-one correspondence between tuples of the desired
	form and selections with replacement of $k-1$ gaps from the $n + 1$
	available gaps, or equivalently, $(k - 1)$-element multisets drawn from a
	set of size $n+1$. Observe that the desired arrangements consist of $n+k-1$
	($n$ stars and $k-1$ bars). Choosing the positions of stars leaves exactly
	$k-1$ spots for $k-1$ bars. In other words, choosing the position of stars
	determines the entire arrangement, so by \cref{theorem2.1.4}, the
	arrangement is determined with $\binom{n+k-1}{n}$ selections. But,
	\begin{equation*}
		\binom{n+k-1}{n} = \frac{(n+k-1)!}{n!(n-n+k-1)!}
		=\frac{(n+k-1)!}{(n+k-1-(k-1))!(k-1)!} = \binom{n+k-1}{k-1}
	\end{equation*}
	reflecting on the fact that choosing the positions for $k-1$ bars also
	determines the arrangements.
\end{proof}
\end{document}
