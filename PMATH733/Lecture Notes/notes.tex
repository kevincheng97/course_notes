\documentclass[11pt]{article}
\usepackage{notes}
\usepackage{faktor}

\theoremstyle{theorem}
\newtheorem{axiom}[theorem]{Axiom}

\newcommand{\thiscoursecode}{PMATH 733}
\newcommand{\thiscoursename}{Set and Model Theory}
\newcommand{\thisprof}{Rahim Moosa}
\newcommand{\me}{Kevin Cheng}
\newcommand{\thisterm}{Fall 2018}

\hypersetup
{pdfauthor={\me},
pdftitle={\thiscoursecode \thisterm Lecutre Notes},
pdfsubject={\thiscoursename}
pdflang={English}}

\begin{document}
\begin{titlepage}
    \begin{centering}
        {\scshape\LARGE University of Waterloo \par}
        \globe
        {\huge\bf \thiscoursecode}\\
        {\scshape\Large \thiscoursename}\\
        \vspace{.3cm}
        {\scshape Prof. \thisprof~\textbullet~\thisterm\par}
    \end{centering}
\sectionline
\tableofcontents
\sectionline
\thispagestyle{empty}

\pagebreak
\section{Introduction}
\subsection{Contact Information}
\begin{itemize}
    \item {\bf Instructor:} Rahim Moosa
    \item Email: rmoosa@uwaterloo.ca
    \item Phone: 519-888-4567, ext. 32453
\\
    \item {\bf Teaching Assistant:} Micheal Deveau
    \item Email: m2deveau@uwaterloo.ca
    \item Phone: 519-888-4567, ext. 32568
    \item Office: MC 5019
\end{itemize}

\subsection{Office Hours}
Every Wednesday, 11:00-12:00 in MC 5018

\subsection{Course Description}
This course is approximately $\faktor{1}{3}$ set theory and $\faktor{2}{3}$
model theory. The set theory will be somewhat naive (i.e., not formal) and the
model theory with be semantic (i.e., no proof theory). There will be some
overlap with PMATH 432/632 (First Order Logic and Computability), but this
latter course is neither a pre-requisite nor an anti-requisite. The
pre-requisite for this course is a familiarity with algebra: groups, fields,
vector spaces, polynomial rings.

Official lecture notes can be purchased for \$13.62 from Media.doc in MC 2018.
The following reference books are on reserve in the Davis Center Library:
\begin{itemize}
	\item \underline{Set theory: an introduction to independence proofs} by
		K. Kunen, and
	\item \underline{Model Theory: an introduction} by D. Marker.
\end{itemize}
Here are the topics I hope to cover:
\begin{itemize}
	\item {\it Set Theory} (four+$\epsilon$ weeks): Zermelo-Fraenkel axioms,
		classes, trans-finite induction/recursion, well-orderings and ordinals,
		the axiom of choice and equivalents, cardinal arithmetic.
	\item {\it Model theory} (eight-$\epsilon$ weeks): First-order logic
		(structures, languages, theories), definable sets, the compactness
		theorem (via ultraproducts) and its consequences, quantifier
		elimination, algebraic examples (vector spaces, algebraically closed
		fields, real closed fields), model companions.
\end{itemize}
There will be five or six homework assignments worth a total of 30\% and a final
exam worth 70\%. There will be no midterm exam.

\pagebreak
\section{Zermelo–Fraenkel Set Theory}
\subsection{Ordinals}
We use the natural numbers $0,1,2,\dots$ to count finite sets. Here, count has
two related meanings:

\begin{itemize}
	\item enumerate, order
	\item measuring size
\end{itemize}

We wish to develop an infinitary generalization of the natural numbers so that
we can $\underbrace{\text{enumberate}}_{\text{ordinals}}$ and
$\underbrace{\text{measure}}_{\text{cardinals}}$ arbitrary sets. But first, we
want a concrete construction of the natural numbers. We need to start somewhere:

\begin{itemize}
	\item set
	\item equality
	\item membership (denotes $\in$)
\end{itemize}

We can construct natural numbers as follows:

\begin{itemize}
	\item $\emptyset = 0 \coloneqq$ the empty set, denoted $\emptyset$ (the set
		with no members)
	\item $\{\emptyset\} = 1 \coloneqq$ the set whose only number is 0, denoted
		by $\{0\}$
	\item $\{\emptyset, \{\emptyset\}\} = 2 \coloneqq$ the set whose only
		members are 0 and 1, so $\{0, 1\}$
	\item and so on...
\end{itemize}

In general, given a natural number $n$ already constructed we deine the next
natural number, the \underline{successor of $n$} to be

\begin{equation*}
	S(n) = n \cup {n}
\end{equation*}

Before continuing, we should ask do these sets exist? Use axioms - unproved
statements.

\begin{axiom}[Empty Set Axiom]
	There exists a set with no members, denoted $\emptyset$.
\end{axiom}

To get 1 from 0, we invoke the following:

\begin{axiom}[Pairset Axiom]
	Given sets $x,y$, there exists a set, denoted by $\{x,y\}$, whose only
	members are $x$ & $y$. That is, \box{$t\in\{x,y\} \iff t = x$ or $t = y$}.
\end{axiom}

If $x=y$, then $\{x,y\}$ has only $x$ as a member. We recognize that $1 =
\{0,0\}$ (denoted by $\{0\}$). This requires an axiom.

\begin{axiom}[Axiom of Extension]
	Given sets $x,y$, $x=y$ \underline{if and only if} $x$ \& $y$ have the same
	members.
\end{axiom}

So far, 0 ($\emptyset$), 1 ($\{\emptyset\}$) and 2 ($\{\emptyset,
\{\emptyset\}\}$) exist. What about $3 \coloneqq \{0,1,2\} =
\underbrace{\{0,1\}}}_{=2}\cup \{2\}$? In general from $n$, to get $S(n) = n
\cup \{n\}$, we need \underline{unions}, given by the following axiom.

\begin{axiom}[Unionset Axiom] Given a set $x$, there exists a set, denoted by
	$Ux$, whose members are precisely the members of the members of $x$. That
	is,
	\begin{equation*}
		t \in Ux \iff t \in y
	\end{equation*}
	for some $y \in x$.
\end{axiom}

So given $n$, $S(n) \coloneqq U\{n,\{n\}\}$. Be definition, $t \in S(n) \iff t
\in n$ or $t = n$. With these axioms, we have constructed rigorously each
natural number. But, what about the set of \underline{all} natural numbers? 

The set of natural numbers should be the smallest set that contains 0 and is
preserved by the successor function. We wish to express this set with definite
conditions.

\begin{axiom}[Infinity Axiom]
	There exists a set $I$ that contains 0 and is preserved by the successor
	function.
\end{axiom}
We can express this axiom in the following definite logical statement.
\begin{equation*}
	\exists I \big((0 \in I) \wedge \forall x (x \in I \impies (\exists y
	(\underbrace{\forall t (t \in y \iff (t \in x) \wedge (t = x))}_{y = S(x)}
	\wedge y \in I))) \big)
\end{equation*}

However, we wish to construct the minimal successor set and we can try to take
the intersection of all successor sets.
\begin{definition}
	If $x$ and $y$ are sets, a \underline{subset} denoted by $x \subseteq y$
	means that every element of $x$ is an element of $y$.
\end{definition}
We can express $x \subeteq y$ as
\begin{equation*}
	\forall z (z \in x \implies z \in y)
\end{equation*}
\begin{axiom}[Powerset Axiom]
	Given a set $x$, there exists a set $\mathcal{P}(x)$ with the axiomerty that
	\begin{equation*}
		\forall t(t \in \mathcal{P}(x) \iff t \subseteq x)
	\end{equation*}
\end{axiom}
\begin{axiom}[(Bounded) Separation Axiom]
	Given a set $x$ and a definite condition $P$, there exists a set of elements
	are precisely the members of $x$ that satisfy $P$.
\end{axiom}
\begin{equation*}
	\exists \{z\in x : P(z)\}
\end{equation*}
\begin{remark}
	It is necessary for us to have (bounded by) $x$ and $P$ is definite.
\end{remark}
\begin{equation*}
	\forall x \exists y \forall t (t \in y \iff (t \in x \wedge P(t)))
\end{equation*}
\begin{exercise}
	Given a non-empty set $x$, there exists a set $\cap x$ satisfying:
	\begin{equation*}
		\forall t (t \in \cap x \iff \forall y (y \in x \implies t \in y))
	\end{equation*}
\end{exercise}
\begin{definition}
	Fix a successor set $I$. The \underline{set of natural numbers} is
	\begin{align*}
		\omega &\coloneqq \cap\{ J \subseteq I : \underbrace{0 \in J \wedge
		\forall x(x \in J \implies S(c) \in J)}_{J \text{ is a successor set
		}}\}\\
		&= \cap\{ J \in \mathcal{P}(I) : J \text{ is a successor set}\}
	\end{align*}
\end{definition}
\begin{exercise}
	This does not depend on $I$.
\end{exercise}
Another very useful axiom that will be used later:
\begin{axiom}[Replacement Axiom]
	Suppose $P$ is a binary definite condition such that for every set $x$ there
	is a unique $y$ satisfying $P(x,y)$. Given a set $A$ there is a set $B$ such
	that $t \in B$ \underline{if and only if} there is an $a \in A$ with
	$P(a, t)$.
\end{axiom}

\subsection{Classes}
We sometimes want to consider collection of sets that do not form a set
themselves. For example, there is no set containing all sets.
\begin{proof}
	Suppose a set $U$ contains all sets. Consider
	\begin{equation*}
		R = \{x \in U : x \notin x\}.
	\end{equation*}
	If $R \in R$, since $R \in U$, $R \notin R$, so a contradiction. But if
	$R \notin R$, by definition $R \in R$, so no such $U$ exists.
\end{proof}

\subsection{Cartiasian Products \& Functions}
\begin{definition}
    Given sets $x,y$, the \underline{ordered pair of $x$ and $y$} is
    \begin{equation*}
        (x,y) = \{\{x\},\{x,y\}\}.
    \end{equation*}
\end{definition}
\begin{definition}
	Given two classes $X,Y$, the \underline{cartisian product of $X$ and $Y$},
	denoted by $X\times Y$ is
	\begin{equation*}
		X\times Y \coloneqq [[z:z=(x,y), x \in X, y\in Y]].
	\end{equation*}
\end{definition}
To see that $X\times Y$ exists, we define the cartisian product with a definite
statement.
\begin{equation*}
	\exists x,y \bigg((x\in X) \wedge (\y\in Y) \wedge \exists a,b
		\underbrace{\forall t (t \in a \iff t = x)}_{a = \{x\}} \wedge
		\underbrace{\forall t(t \in b \iff t = x \vee t=y)}_{b=\{x,y\}}\wedge
	\underbrace{\forall(t\in z\ iff t\ =z \vee t=b)}_{z = \{a,b\}}\bigg)
\end{equation*}
\begin{remark}
	If $A$ is a set, $B$ is a class and $B\subseteq A$, then $B$ is a set.
\end{remark}
\begin{proof}
	Notice,
	\begin{equation*}
		B = \{a\in A : a \in B\}
	\end{equation*}
	So by (bounded) separation $B$ is a set.
\end{proof}
\begin{remark}
	If $X,Y$ are sets the $\mathcal{P}(\mathcal{P}(X\cup Y))$ is a set and
	\begin{equation*}
		X\times Y \subseteq \mathcal{P}(\mathcal{P}(X\cup Y))
	\end{equation*}
	so $X\times Y$ is a set.
\end{remark}
\begin{definition}
	Given classes $X,Y$, a \underline{definite operation} $f:X\to Y$ is a class
	$\Gamma(f)$ such that $\Gamma(f)\subseteq X\times Y$ and for every $x \in X$
	there is a unique $y \in Y$ such that $(x,y) \in \Gamma(f)$. We write $f(x)
	= y$ to mean that $(x,y) \in \Gamma{f}$
\end{definition}
\begin{remark}
	If $X,Y$ are sets and $f:X\to Y$ is a definite operation, then $\Gamma(f)
	\subseteq \underbrace{X \times Y}_{\text{set}}$, so $\Gamma(f)$ is a set.
	In this case, we call $f$ a \underline{function}.
\end{remark}
\begin{definition}
	A \underline{function} is a definite operations $f:X\to Y$ where $X,Y$ are
	both sets.
\end{definition}
\begin{prop}[Induction Principle]
	Suppose $J \subseteq \omega$, $0 \in J$ and whenever $n \in J$ then $S(n)
	\in J$. Then $J = \omega$.
\end{prop}
\begin{proof}
	$J$ is a successor set, so by definition, $\omega \subseteq J$, so $J =
	\omega$.
\end{proof}
\begin{lemma}
	Suppose $n \in \omega$.
	\begin{enumerate}[label=(\alph*)]
		\item Every element of $n$ is a subset of $\omega$.
		\item Every element of $n$ is a subset of $n$.
		\item $n \notin n$.
		\item Either $n = 0$ or $0 \in n$.
		\item If $y \in n$ then either $S(y) \in n$ or $S(y) = n$.
	\end{enumerate}
\end{lemma}
\end{document}
