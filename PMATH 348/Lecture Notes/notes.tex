\documentclass[11pt]{article}
\usepackage{notes}
\usepackage{faktor}

\newcommand{\thiscoursecode}{PMATH 352}
\newcommand{\thiscoursename}{Fields and Galois Theory}
\newcommand{\thisprof}{Prof. Yu-Ru Liu}
\newcommand{\me}{Kevin Cheng}
\newcommand{\thisterm}{Winter 2018}

\newcommand{\cyclic}[1]{\langle #1 \rangle}
\newcommand{\quotient}[2]{\faktor{#1}{#2}}

\hypersetup
{pdfauthor={\me},
pdftitle={\thiscoursecode \thisterm Lecutre Notes},
pdfsubject={\thiscoursename}
pdflang={English}}

\begin{document}

\begin{titlepage}
\begin{centering}
{\scshape\LARGE University of Waterloo \par}
\globe
{\huge\bf PMATH 348}\\
{\scshape\Large Fields and Galois Theory}\\
\vspace{.3cm}
{\scshape Prof. Yu-Ru Liu~\textbullet~Winter 2018\par}
\end{centering}
\sectionline
\tableofcontents
\sectionline
\thispagestyle{empty}
\end{titlepage}

\section{Introduction}
\subsection{Polynomial Equations}
Consider the quadratic equation. Let $ax^2 + bx + c = 0$ with the leading
coefficient $a \neq 0$, then we have that,
\begin{equation*}
x = \frac{-b \pm \sqrt{b^2 - 4ac}}{2a}
\end{equation*}
We notice immediately that there are a couple of operations that are involved in
this equation.
\begin{definition}
An expression involving only addition, subtraction, multiplication, division and
radicals is called a \underline{radical}. These operations are denoted by $+, -, \times,
\div$ and $\sqrt[n]{\cdot}$
\end{definition}
The natural question that is raised is the extension to higher dimensions.
\subsection{Cubic Equations}
All cubic equations can be reduced to the following equation,
\begin{align*}
x^3 + px = q
\end{align*}
for some $p, q \in \C$. A solution to the above equation is of the form
\begin{equation*}
x = \sqrt[3]{\frac{q}{2} + \sqrt{\frac{p^3}{27} + \frac{q^2}{4}}} +
\sqrt[3]{\frac{q}{2} - \sqrt{\frac{p^3}{27} + \frac{q^2}{4}}}\tag{Cardano's
Formula}
\end{equation*}
\subsection{Quartic Equations}
A radical solution can be obtained by reducing a quartic to a cubic equation.
\subsection{Quintic Equations}
\begin{itemize}
\item General radical solutions were attempted by Euler, B\'ezout and Lagrange without
success
\item In 1799, Ruffini gave a 516 page proof about the insolvability of quintic
equations. His Proof was ``almost right''
\item In 1824, Abel filled the gap in Ruffini's proof.
\end{itemize}
We can now ask ourselves, given a quintic equation, is it solvable by radicals?
This question seems to be too hard, so we ask, suppose that a radical solution
exists. How does its associated quintic equation look like?\\

\underline{Two main steps in Galois Theory}
\begin{enumerate}
\item Link a root of a quintic equation, say $\alpha$ to $\Q(\alpha)$, the
smallest field containing $\Q$ and $\alpha$. $\Q(\alpha)$ is a field. So it has
more structures to be played with than $\alpha$; however, our knowledge of
$\Q(\alpha)$ is still too little to answer the question. For example, we do not
know how many intermediate fields, $E$ between $\Q$ and $\Q(\alpha)$.
What we mean is how many fields $E$ satisfy
\begin{equation*}
\Q \subseteq E \subseteq \Q(\alpha).
\end{equation*}
\item Link the field $\Q(\alpha)$ to a group. More precisely, we associate
$\Q(\alpha)/\Q$ to the group
\begin{equation*}
\Aut_\Q (\Q(\alpha)) = \bigg\{\Psi : \Q(\alpha) \to \Q(\alpha)\>\text{an
isomorphism and } \Psi|_\Q = 1_\Q\bigg\}
\end{equation*}
It can be shown that if $\alpha$ is ``good'', say algebraic, $\Aut_\Q
(\Q(\alpha))$ is finite. If $\alpha$ is ``very good'', say constructable, the order
of $\Aut_\Q (\Q(\alpha))$ is in certain forms. Moreover, there is a one-to-one
correspondence between the intermediate fields between $\Q(\alpha)$ and
$\Q$ and the subgroups of $\Aut_\Q (\Q(\alpha))$. \\
\end{enumerate}

It follows that given some ``good'' $\alpha$, we have that the
intermediate fields of $\Q(\alpha)$ and $\Q$ are indeed finitely many. This
introduces \underline{Galois Theory}; the interplay between fields and groups.

\section{Field Extensions}
\subsection{Degree of Extensions}
\begin{definition}
If $E$ is a field containing another field $F$, we say $E$ is a
\underline{field extension} of $F$, denoted by $\quotient EF$.
\end{definition}
If $\quotient{E}{F}$ if a field extension, we can view $E$ as a vector space over $F$.
\begin{enumerate}
\item \underline{Addition}: For $e_1, e_2 \in E$, $e_1 + e_2 \coloneqq e_1 +
e_2$ (addition in $E$)
\item \underline{Scalar Multiplication}: For $c \in F, e \in E$, $c \cdot e
\coloneqq ce$ (multiplication in $E$)
\end{enumerate}
\begin{definition}
The dimension of $E$ over $F$ (viewed as a vector space) called the
\underline{degree} of $E$ over $F$, denoted by $[E:F]$. If $[E:F] < \infty$, we
say $\quotient EF$ is a \underline{finite extension}. Otherwise, $\quotient{E}{F}$ is an
\underline{infinite extension}
\end{definition}
\begin{example}
$[\C : \R] = 2$ is a  finite extension since $\C \cong \R + \R i$, with $i^2 =
-1$.
\end{example}
\begin{example}
Let $F$ be a field. Then $[F(x) : F]$ is $\infty$ since ${1, x, x^2, \dots}$ are
linearly independent over $F$.
\end{example}
\begin{remark}
$F[x] = \{f(x) = a_0 + a_1x + \dots + a_nx^n : a_i \in F, n \in \N \cup
\{0\}\}$
\end{remark}
\begin{remark}
$F(x) = \{\frac{f(x)}{g(x)} : f(x), g(x) \in F[x]\}$
\end{remark}
\begin{theorem}
The $E/K$ and $K/F$ are finite field extensions, then $\quotient{E}{F}$ is a finite field
extension and
\begin{equation*}
[E:F] = [E:K][K:F]
\end{equation*}
In particular, $K$ is an intermediate field of an field extension $\quotient{E}{F}$, then
$[K:F]\>\big|\> [E:F]$. 
\end{theorem}
\begin{proof}
Suppose $[E:K]=m$ and $[K:F] = n$. Let $\{a_i,\dots,a_m\}$ be a basis of
$E/K$ and $\{b_1, \dots, b_n\}$ be a basis of $K/F$. It suffices to show
$\{a_ib_j:1 \leq i \leq m, 1 \leq j \leq n\}$ is a basis of $[\quotient{E}{F}]$.\\

\begin{claim}
Every element of $E$ is a linear combination of $\{a_ib_j\}$ over $F$.
\end{claim}

For $e \in E$, we have
\begin{equation*}
e = \sum^m_{i=1} k_ia_i 
\end{equation*}
with $k_i \in K$. Also, for each $k_i \in K$, we have
\begin{equation*}
k_i = \sum^n_{j=1} c_{ij}b_j 
\end{equation*}
with $c_{ij} \in F$. Thus,
\begin{equation*}
e = \sum^m_{i=1} \sum^n_{j=1} c_{ij}b_ja_i.
\end{equation*}

\begin{claim}
The set $\{a_ib_j:1\leq i\leq m,1\leq j\leq n\}$ is linearly independent
over $F$.
\end{claim}
Suppose that $$\sum_{i=1}^m\sum_{j=1}^nc_{ij}b_ja_i=0$$ with $c_{ij}\in F$. Since
$\sum_{j=1}^nc_{ij}b_j\in K$ and $\{a_1,\dots,a_m\}$ are independent over $K$.
We have $$\sum_{j=1}^nc_{ij}b_j=0.$$
Since $\{b_1,\dots,b_n\}$ are independent over $F$, we have $c_{ij}=0$.\\

Combining both claims, we see that $\{a_ib_j,1\leq i\leq m,1\leq j\leq n\}$ is a
basis of $\quotient{E}{F}$ and we have $[E:F]=[E:K][K:F]$.
\end{proof}

\subsection{Algebraic and Transcendental Extensions}
\begin{definition}
Let $\quotient{E}{F}$ be a field extension and $\alpha \in E$. We say $\alpha$ is
\underline{algebraic over $F$} if there exists $f(x) \in F[x]\setminus \{0\}$
with $f(\alpha) = 0$. Otherwise, $\alpha$ is \underline{transcendental over
$F$}.
\end{definition}
\begin{example}
$\frac{c}{d} \in \Q$, $\sqrt 2$ $\sqrt[2]{7} + 2i$ are algebraic over $\Q$
(see Assignment 1) but $e$ (Hermite, 1873) and $\pi$ (Lindemann, 1882) are
transcendental over $\Q$.
\end{example}
Let $\quotient{E}{F}$ be a field extension and $\alpha \in E$. Let $F[\alpha]$ denote the
smallest subfield of $E$ containing $F$ and $\alpha$. For $\alpha, \beta \in E$,
we define $F[\alpha, \beta]$ and $F(\alpha, \beta)$ similarly.
\begin{definition}
If $F = F(\alpha)$ for some $\alpha \in E$, we say $E$ is a \underline{simple
extension} of $F$.
\end{definition}
\begin{definition}
Let $R_1$ and $R_2$ be two rings which contain a field $F$. A ring homomorphism
$\Psi: R_1 \to R_2$ is said to be a \underline{$F$-homomorphism} if $\Psi|_F =
1_F$.
\end{definition}
\begin{theorem}
Let $\quotient{E}{F}$ be a field extension and $\alpha \in E$. If $\alpha$ is transcendental
over $F$, then
\begin{equation*}
F[\alpha] \cong F[x] \quad \text{and} \quad F(\alpha) \cong F(x)
\end{equation*}
In particular, $F[\alpha] \neq F(\alpha)$.
\end{theorem}
\begin{remark}
In fact, if $\alpha$ is algebraic, indeed $F[\alpha] = F(\alpha)$.
\end{remark}
\begin{proof}
Let $\Psi:F(x) \to F(\alpha)$ be the unique $F$-homomorphism defined by $\Psi(x)
= \alpha$. Thus, for $f(x), g(x) \in F[x], g(x) \neq 0$, 
$$\Psi\bigg(\frac{f(x)}{g(x)}\bigg) = \frac{f(\alpha)}{g(\alpha)} \in
F(\alpha).$$ 
Notice that this is indeed a well-defined map as $g(x) \neq 0$ implies
$g(\alpha) \neq 0$ since $\alpha$ is transcendental. Since $F(x)$ is a field and
$\ker(\Psi)$ is an ideal of $F(x)$, we have $\ker(\Psi) = F(x)$ or trivial. This
$\Psi = 0$ or $\Psi$ is injective. Since $\Psi(x) = \alpha \neq 0$, $\Psi$ must
be injective. Also, since $F(x)$ is a field, $\Ima\Psi$ contains a field
generated by $F$ and $\alpha$, in other words, $F(\alpha) \subseteq \Ima\Psi$.
Thus, $\Ima\Psi = F(\alpha)$ and $\Psi$ is surjective. It follows that
$\Psi$ is an isomorphism and we have
\begin{equation*}
F[\alpha] \cong F[x] \quad \text{and} \quad F(\alpha) \cong F(x)
\end{equation*}
\end{proof}
\begin{theorem}
Let $\quotient{E}{F}$ be a field extension and $\alpha \in E$. If $\alpha$ is algebraic over
$F$, there exists a unique monic irreducible polynomial $p(x) \in F[x]$ such
that there exists a $F$-homomorphism
\begin{equation*}
\Psi:\quotient{F[x]}{\cyclic{p(x)}} \to F[\alpha] \quad \text{with } \Psi(x) = \alpha
\end{equation*}
from which we conclude $F[\alpha] \cong F(\alpha)$.
\end{theorem}
\begin{proof}
Consider the unique $F$-homomorphism $\Psi: F[x] \to F[\alpha]$ defined by
$\Psi(x) = \alpha$. Thus, for $f(x) \in F[x]$, we have $\Psi(f) = f(\alpha)$.
Since $F[x]$ is a ring, $\Ima\Psi$ contains a ring generated by $F$ and
$\alpha$, in other words, $F[\alpha] \subseteq \Ima\Psi$. Thus, $\Ima\Psi =
f[\alpha]$.\\
Let
\begin{equation*}
I = \Ker\Psi = \{f(x) \in F[x] : f(\alpha) = 0 \}.
\end{equation*}
Since $\alpha$ is algebraic, $I \neq \{0\}$. We have $\quotient{F[x]}{I} \cong
\Ima\Psi = F[\alpha] \subseteq F(\alpha)$, a subring of a field $F(\alpha)$. Thus,
$\quotient{F[x]}{I}$ is an integral domain so $I$ is a prime ideal. It follows that
$I = \cyclic{p(x)}$, where $p(x)$ is irreducible. If we assume $p(x)$ is
monic, then it is unique. It follows that
\begin{equation*}
\quotient{F[x]}{\cyclic{p(x)}}\cong F[\alpha].
\end{equation*}
Since $p(x)$ is irreducible, $\quotient{F[x]}{\cyclic{p(x)}}$ is a field. So
$F[\alpha]$ is a field. It follows that $F[\alpha] = F(\alpha)$.
\end{proof}
\begin{definition}
If $\alpha$ is algebraic over a field $F$, the unique monic polynomial
irreducible polynomial $p(x)$ in Theorem 3 is called the \underline{minimal
polynomial of $\alpha$ over $F$}.
\end{definition}
\begin{remark}
From the proof of Theorem 3, if $f(x) \in F[x]$ with $f(\alpha) = 0$, then
$p(x) \big| f(x)$.
\end{remark}
\begin{theorem}
Let $\quotient{E}{F}$ be a field extension and $\alpha \in E$.
\begin{enumerate}
\item $\alpha$ is transcendental over $F$ if and only if $[F(\alpha):F]$ is
$\infty$.
\item $\alpha$ is algebraic over $F$ if and only if $[F(\alpha):F] < \infty$.
\end{enumerate}
Moreover, if $p(x)$ is the minimal polynomial of $\alpha$ over $F$, we have
$[F[\alpha]:F] = \deg(p)$ and $\{1, \alpha, \alpha^2, \dots,
\alpha^{\deg (p) - 1}\}$ is a basis of $\quotient{F(\alpha)}{F}$.
\end{theorem}
\begin{proof}
It suffices to prove the forward direction for each statement as the inverse
direction implies the other statement.\\
(1) {\bf Forwards}: From Theorem 2, if $\alpha$ is transcendental over $F$, then
$F(x) \cong F(\alpha)$. In $F(x)$, the elements $\{1, x, x^2, \dots\}$ are
linearly independent over $F$. Thus, $[F(\alpha):F]$ is $\infty$.\\
(2) {\bf Forwards}: From Theorem 3, if $\alpha$ is algebraic over $F$,
$\quotient{F[x]}{\cyclic{p(x)}} \cong F(x)$ with the map $x \mapsto \alpha$. Note
that,
\begin{equation*}
\quotient{F[x]}{\cyclic{p(x)}} \cong \{r(x) \in F[x]: \deg(r) < \deg(p)\}
\tag{$\deg(0) = -\infty$}
\end{equation*}
Thus, $\{1, x, x^2, \dots, x^{\deg(p) - 1}\}$ forms a basis for
$\quotient{F[x]}{\cyclic{p(x)}}$. It follows that $[F(\alpha): F] =
\deg(p)$ and $\{1, \alpha, \alpha^2, \dots,
\alpha^{\deg (p) - 1}\}$ is a basis of $\quotient{F(\alpha)}{F}$.
\end{proof}
\begin{theorem}
Let $\quotient{E}{F}$ be a field extension. If $[E:F] < \infty$, then there exists $\alpha_1, \dots, \alpha_n \in E$ such that
\begin{equation*}
F \subsetneq F(\alpha_1) \subsetneq \dots \subsetneq F(\alpha_1, \dots, \alpha_n) = E.
\end{equation*}
\end{theorem}
\end{document}
