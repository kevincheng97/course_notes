\documentclass[11pt]{article}
\usepackage{notes}
\usepackage{faktor}

\newcommand{\thiscoursecode}{PMATH 348}
\newcommand{\thiscoursename}{Fields and Galois Theory}
\newcommand{\thisprof}{Yu-Ru Liu}
\newcommand{\me}{Kevin Cheng}
\newcommand{\thisterm}{Winter 2018}

\newcommand{\cyclic}[1]{\left\langle #1 \right\rangle}
\newcommand{\quotient}[2]{\faktor{#1}{#2}}
\newcommand{\Mod}[1]{\ (\mathrm{mod}\ #1)}
\DeclareMathOperator{\Frac}{Frac}
\DeclareMathOperator{\ch}{ch}

\hypersetup
{pdfauthor={\me},
pdftitle={\thiscoursecode \thisterm Lecutre Notes},
pdfsubject={\thiscoursename}
pdflang={English}}

\begin{document}

\begin{titlepage}
\begin{centering}
{\scshape\LARGE University of Waterloo \par}
\globe
{\huge\bf \thiscoursecode}\\
{\scshape\Large \thiscoursename}\\
\vspace{.3cm}
{\scshape Prof. \thisprof~\textbullet~\thisterm\par}
\end{centering}
\sectionline
\tableofcontents
\sectionline
\thispagestyle{empty}
\end{titlepage}

\section{Introduction}
\subsection{Polynomial Equations}
Consider the quadratic equation. Let $ax^2 + bx + c = 0$ with the leading
coefficient $a \neq 0$, then we have that,
\begin{equation*}
x = \frac{-b \pm \sqrt{b^2 - 4ac}}{2a}
\end{equation*}
We notice immediately that there are a couple of operations that are involved in
this equation.
\begin{definition}
An expression involving only addition, subtraction, multiplication, division and
radicals is called a \underline{radical}. These operations are denoted by $+, -, \times,
\div$ and $\sqrt[n]{\cdot}$
\end{definition}
The natural question that is raised is the extension to higher dimensions.
\subsection{Cubic Equations}
All cubic equations can be reduced to the following equation,
\begin{align*}
x^3 + px = q
\end{align*}
for some $p, q \in \C$. A solution to the above equation is of the form
\begin{equation*}
x = \sqrt[3]{\frac{q}{2} + \sqrt{\frac{p^3}{27} + \frac{q^2}{4}}} +
\sqrt[3]{\frac{q}{2} - \sqrt{\frac{p^3}{27} + \frac{q^2}{4}}}\tag{Cardano's
Formula}
\end{equation*}
\subsection{Quartic Equations}
A radical solution can be obtained by reducing a quartic to a cubic equation.
\subsection{Quintic Equations}
\begin{itemize}
\item General radical solutions were attempted by Euler, B\'ezout and Lagrange without
success
\item In 1799, Ruffini gave a 516 page proof about the insolvability of quintic
equations. His Proof was ``almost right''
\item In 1824, Abel filled the gap in Ruffini's proof.
\end{itemize}
We can now ask ourselves, given a quintic equation, is it solvable by radicals?
This question seems to be too hard, so we ask, suppose that a radical solution
exists. How does its associated quintic equation look like?\\

\underline{Two main steps in Galois Theory}
\begin{enumerate}
\item Link a root of a quintic equation, say $\alpha$ to $\Q(\alpha)$, the
smallest field containing $\Q$ and $\alpha$. $\Q(\alpha)$ is a field. So it has
more structures to be played with than $\alpha$; however, our knowledge of
$\Q(\alpha)$ is still too little to answer the question. For example, we do not
know how many intermediate fields, $E$ between $\Q$ and $\Q(\alpha)$.
What we mean is how many fields $E$ satisfy
\begin{equation*}
\Q \subseteq E \subseteq \Q(\alpha).
\end{equation*}
\item Link the field $\Q(\alpha)$ to a group. More precisely, we associate
$\Q(\alpha)/\Q$ to the group
\begin{equation*}
\Aut_\Q (\Q(\alpha)) = \bigg\{\Psi : \Q(\alpha) \to \Q(\alpha)\>\text{an
isomorphism and } \Psi|_\Q = 1_\Q\bigg\}
\end{equation*}
It can be shown that if $\alpha$ is ``good'', say algebraic, $\Aut_\Q
(\Q(\alpha))$ is finite. If $\alpha$ is ``very good'', say constructable, the order
of $\Aut_\Q (\Q(\alpha))$ is in certain forms. Moreover, there is a one-to-one
correspondence between the intermediate fields between $\Q(\alpha)$ and
$\Q$ and the subgroups of $\Aut_\Q (\Q(\alpha))$. \\
\end{enumerate}

It follows that given some ``good'' $\alpha$, we have that the
intermediate fields of $\Q(\alpha)$ and $\Q$ are indeed finitely many. This
introduces \underline{Galois Theory}; the interplay between fields and groups.

\pagebreak
\section{Field Extensions}
\subsection{Degree of Extensions}
\begin{definition}
If $E$ is a field containing another field $F$, we say $E$ is a
\underline{field extension} of $F$, denoted by $\quotient EF$.
\end{definition}
If $\quotient{E}{F}$ if a field extension, we can view $E$ as a vector space over $F$.
\begin{enumerate}
\item \underline{Addition}: For $e_1, e_2 \in E$, $e_1 + e_2 \coloneqq e_1 +
e_2$ (addition in $E$)
\item \underline{Scalar Multiplication}: For $c \in F, e \in E$, $c \cdot e
\coloneqq ce$ (multiplication in $E$)
\end{enumerate}
\begin{definition}
The dimension of $E$ over $F$ (viewed as a vector space) called the
\underline{degree} of $E$ over $F$, denoted by $[E:F]$. If $[E:F] < \infty$, we
say $\quotient EF$ is a \underline{finite extension}. Otherwise, $\quotient{E}{F}$ is an
\underline{infinite extension}
\end{definition}
\begin{example}
$[\C : \R] = 2$ is a  finite extension since $\C \cong \R + \R i$, with $i^2 =
-1$.
\end{example}
\begin{example}
Let $F$ be a field. Then $[F(x) : F]$ is $\infty$ since $\{1, x, x^2, \dots\}$ are
linearly independent over $F$.
\end{example}
\begin{remark}
$F[x] = \{f(x) = a_0 + a_1x + \dots + a_nx^n : a_i \in F, n \in \N \cup
\{0\}\}$, the polynomial ring of $F$.
\end{remark}
\begin{remark}
$F(x) = \{\frac{f(x)}{g(x)} : f(x), g(x) \in F[x]\}$, the fraction
field of the polynomial ring of $F$. 
\end{remark}
\begin{theorem} \label{theorem1}
If $E/K$ and $K/F$ are finite field extensions, then $\quotient{E}{F}$ is a finite field
extension and
\begin{equation*}
[E:F] = [E:K][K:F]
\end{equation*}
In particular, $K$ is an intermediate field of an field extension $\quotient{E}{F}$, then
$[K:F]\>\big|\> [E:F]$. 
\end{theorem}
\begin{proof}
Suppose $[E:K]=m$ and $[K:F] = n$. Let $\{a_i,\dots,a_m\}$ be a basis of
$E/K$ and $\{b_1, \dots, b_n\}$ be a basis of $K/F$. It suffices to show
$\{a_ib_j:1 \leq i \leq m, 1 \leq j \leq n\}$ is a basis of $\quotient{E}{F}$.\\

\begin{claim}
Every element of $E$ is a linear combination of $\{a_ib_j\}$ over $F$.
\end{claim}

For $e \in E$, we have
\begin{equation*}
e = \sum^m_{i=1} k_ia_i 
\end{equation*}
with $k_i \in K$. Also, for each $k_i \in K$, we have
\begin{equation*}
k_i = \sum^n_{j=1} c_{ij}b_j 
\end{equation*}
with $c_{ij} \in F$. Thus,
\begin{equation*}
e = \sum^m_{i=1} \sum^n_{j=1} c_{ij}b_ja_i.
\end{equation*}

\begin{claim}
The set $\{a_ib_j:1\leq i\leq m,1\leq j\leq n\}$ is linearly independent
over $F$.
\end{claim}
Suppose that $$\sum_{i=1}^m\sum_{j=1}^nc_{ij}b_ja_i=0$$ with $c_{ij}\in F$. Since
$\sum_{j=1}^nc_{ij}b_j\in K$ and $\{a_1,\dots,a_m\}$ are independent over $K$.
We have $$\sum_{j=1}^nc_{ij}b_j=0.$$
Since $\{b_1,\dots,b_n\}$ are independent over $F$, we have $c_{ij}=0$.\\

Combining both claims, we see that $\{a_ib_j,1\leq i\leq m,1\leq j\leq n\}$ is a
basis of $\quotient{E}{F}$ and we have $[E:F]=[E:K][K:F]$.
\end{proof}

\subsection{Algebraic and Transcendental Extensions}
\begin{definition}
Let $\quotient{E}{F}$ be a field extension and $\alpha \in E$. We say $\alpha$ is
\underline{algebraic over $F$} if there exists $f(x) \in F[x]\setminus \{0\}$
with $f(\alpha) = 0$. Otherwise, $\alpha$ is \underline{transcendental over
$F$}.
\end{definition}
\begin{example}
$\frac{c}{d} \in \Q$, $\sqrt 2$ $\sqrt[3]{7} + 2i$ are algebraic over $\Q$
(see Assignment 1) but $e$ (Hermite, 1873) and $\pi$ (Lindemann, 1882) are
transcendental over $\Q$.
\end{example}
Let $\quotient{E}{F}$ be a field extension and $\alpha \in E$. Let $F[\alpha]$ denote the
smallest subfield of $E$ containing $F$ and $\alpha$. For $\alpha, \beta \in E$,
we define $F[\alpha, \beta]$ and $F(\alpha, \beta)$ similarly.
\begin{definition}
If $F = F(\alpha)$ for some $\alpha \in E$, we say $E$ is a \underline{simple
extension} of $F$.
\end{definition}
\begin{definition}
Let $R_1$ and $R_2$ be two rings which contain a field $F$. A ring homomorphism
$\Psi: R_1 \to R_2$ is said to be a \underline{$F$-homomorphism} if $\Psi|_F =
1_F$.
\end{definition}
\begin{theorem} \label{theorem2}
Let $\quotient{E}{F}$ be a field extension and $\alpha \in E$. If $\alpha$ is transcendental
over $F$, then
\begin{equation*}
F[\alpha] \cong F[x] \quad \text{and} \quad F(\alpha) \cong F(x)
\end{equation*}
In particular, $F[\alpha] \neq F(\alpha)$.
\end{theorem}
\begin{remark}
In fact, if $\alpha$ is algebraic, indeed $F[\alpha] = F(\alpha)$.
\end{remark}
\begin{proof}
Let $\Psi:F(x) \to F(\alpha)$ be the unique $F$-homomorphism defined by $\Psi(x)
= \alpha$. Thus, for $f(x), g(x) \in F[x], g(x) \neq 0$, 
$$\Psi\bigg(\frac{f(x)}{g(x)}\bigg) = \frac{f(\alpha)}{g(\alpha)} \in
F(\alpha).$$ 
Notice that this is indeed a well-defined map as $g(x) \neq 0$ implies
$g(\alpha) \neq 0$ since $\alpha$ is transcendental. Since $F(x)$ is a field and
$\ker(\Psi)$ is an ideal of $F(x)$, we have $\ker(\Psi) = F(x)$ or trivial. Thus
$\Psi = 0$ or $\Psi$ is injective. Since $\Psi(x) = \alpha \neq 0$, $\Psi$ must
be injective. Also, since $F(x)$ is a field, $\im(\Psi)$ contains a field
generated by $F$ and $\alpha$, in other words, $F(\alpha) \subseteq \im(\Psi)$.
Thus, $\im(\Psi) = F(\alpha)$ and $\Psi$ is surjective. It follows that
$\Psi$ is an isomorphism and we have
\begin{equation*}
F[\alpha] \cong F[x] \quad \text{and} \quad F(\alpha) \cong F(x).
\end{equation*}
\end{proof}
\begin{theorem} \label{theorem3}
Let $\quotient{E}{F}$ be a field extension and $\alpha \in E$. If $\alpha$ is algebraic over
$F$, there exists a unique monic irreducible polynomial $p(x) \in F[x]$ such
that there exists a $F$-homomorphism
\begin{equation*}
\Psi:\quotient{F[x]}{\cyclic{p(x)}} \to F[\alpha] \quad \text{with } \Psi(x) = \alpha
\end{equation*}
from which we conclude $F[\alpha] \cong F(\alpha)$.
\end{theorem}
\begin{proof}
Consider the unique $F$-homomorphism $\Psi: F[x] \to F[\alpha]$ defined by
$\Psi(x) = \alpha$. Thus, for $f(x) \in F[x]$, we have $\Psi(f) = f(\alpha)$.
Since $F[x]$ is a ring, $\im(\Psi)$ contains a ring generated by $F$ and
$\alpha$, in other words, $F[\alpha] \subseteq \im(\Psi)$. Thus, $\im(\Psi) =
F[\alpha]$.\\
Let
\begin{equation*}
I = \ker(\Psi) = \{f(x) \in F[x] : f(\alpha) = 0 \}.
\end{equation*}
Since $\alpha$ is algebraic, $I \neq \{0\}$. We have $\quotient{F[x]}{I} \cong
\im(\Psi) = F[\alpha] \subseteq F(\alpha)$, a subring of a field $F(\alpha)$. Thus,
$\quotient{F[x]}{I}$ is an integral domain so $I$ is a prime ideal. It follows that
$I = \cyclic{p(x)}$, where $p(x)$ is irreducible. If we assume $p(x)$ is
monic, then it is unique. It follows that
\begin{equation*}
\quotient{F[x]}{\cyclic{p(x)}}\cong F[\alpha].
\end{equation*}
Since $p(x)$ is irreducible, $\quotient{F[x]}{\cyclic{p(x)}}$ is a field. So
$F[\alpha]$ is a field. It follows that $F[\alpha] = F(\alpha)$.
\end{proof}
\begin{definition}
If $\alpha$ is algebraic over a field $F$, the unique monic polynomial
irreducible polynomial $p(x)$ in \cref{theorem3} is called the \underline{minimal
polynomial of $\alpha$ over $F$}.
\end{definition}
\begin{remark}
From the proof of \cref{theorem3}, if $f(x) \in F[x]$ with $f(\alpha) = 0$, then
$p(x) \big| f(x)$.
\end{remark}
\begin{theorem} \label{theorem4}
Let $\quotient{E}{F}$ be a field extension and $\alpha \in E$.
\begin{enumerate}
\item $\alpha$ is transcendental over $F$ if and only if $[F(\alpha):F]$ is
$\infty$.
\item $\alpha$ is algebraic over $F$ if and only if $[F(\alpha):F] < \infty$.
\end{enumerate}
Moreover, if $p(x)$ is the minimal polynomial of $\alpha$ over $F$, we have
$[F[\alpha]:F] = \deg(p)$ and $\{1, \alpha, \alpha^2, \dots,
\alpha^{\deg (p) - 1}\}$ is a basis of $\quotient{F(\alpha)}{F}$.
\end{theorem}
\begin{proof}
It suffices to prove the forward direction for each statement as the inverse
direction implies the other statement.\\
(1) {\bf Forwards}: From \cref{theorem2}, if $\alpha$ is transcendental over $F$, then
$F(x) \cong F(\alpha)$. In $F(x)$, the elements $\{1, x, x^2, \dots\}$ are
linearly independent over $F$. Thus, $[F(\alpha):F]$ is $\infty$.\\
(2) {\bf Forwards}: From \cref{theorem3}, if $\alpha$ is algebraic over $F$,
$\quotient{F[x]}{\cyclic{p(x)}} \cong F(x)$ with the map $x \mapsto \alpha$. Note
that,
\begin{equation*}
\quotient{F[x]}{\cyclic{p(x)}} \cong \{r(x) \in F[x]: \deg(r) < \deg(p)\}
\tag{$\deg(0) = -\infty$}
\end{equation*}
Thus, $\{1, x, x^2, \dots, x^{\deg(p) - 1}\}$ forms a basis for
$\quotient{F[x]}{\cyclic{p(x)}}$. It follows that $[F(\alpha): F] =
\deg(p)$ and $\{1, \alpha, \alpha^2, \dots,
\alpha^{\deg (p) - 1}\}$ is a basis of $\quotient{F(\alpha)}{F}$.
\end{proof}
\begin{theorem}
Let $\quotient{E}{F}$ be a field extension. If $[E:F] < \infty$, then there exists $\alpha_1, \dots, \alpha_n \in E$ such that
\begin{equation*}
F \subsetneq F(\alpha_1) \subsetneq \dots \subsetneq F(\alpha_1, \dots, \alpha_n) = E.
\end{equation*}
\end{theorem}
\begin{proof}
We proceed with induction on $[E:F]$. If $[E:F] = 1$, $E=F$. Suppose that $[E:F]
> 1$ and the statement holds for any field extension $\quotient{\tilde
E}{\tilde F}$ with $[\tilde E : \tilde F] < [E:F]$. Let $\alpha_1 \in \quotient{E}{F}$.
By \cref{theorem1},
\begin{equation*}
[E:F] = [E:F(\alpha_1)][F(\alpha_1) : F].
\end{equation*}
Since $[F(\alpha):F] > 1$, we have $[E:F] > [E:F(\alpha_1)]$. By induction
hypothesis, there exists $\alpha_2, \dots, \alpha_n$ such that
\begin{equation*}
F(\alpha_1) \subsetneq \dots \subsetneq F(\alpha_1, \dots, \alpha_n) = E.
\end{equation*}
Thus, we have
\begin{equation*}
F \subsetneq F(\alpha_1) \subsetneq \dots \subsetneq F(\alpha_1, \dots, \alpha_n) = E.
\end{equation*}
as desired.
\end{proof}
\begin{definition}
A field extension $\quotient{E}{F}$ is \underline{algebraic} if every $\alpha
\in E$ is algebraic over $F$. Otherwise, it is \underline{transcendental}.
\end{definition}
\begin{theorem} \label{theorem6}
Let $\quotient{E}{F}$ be a field extension. If $[E:F] < \infty$, then
$\quotient{E}{F}$ is algebraic.
\end{theorem}
\begin{proof}
Suppose $[E:F] = n$. For $\alpha \in E$, the elements $\{1, \alpha, \dots,
\alpha^n\}$ are not linearly independent over $F$. Thus, there exists $c_i \in
F$ for all $i = 0, \dots , n$, not all 0, such that
\begin{equation*}
\sum^n_{i = 0} c_i\alpha^i = 0
\end{equation*}
Thus, $\alpha$ is a root of the polynomial $\sum^n_{i = 0} c_i\alpha^i \in
F[x]$ so it is algebraic over $F$.
\end{proof}
\begin{theorem}
Let $\quotient{E}{F}$ be a field extension. Define,
\begin{equation*}
L \coloneqq \{\alpha \in E: [F(\alpha) : F] < \infty\}.
\end{equation*}
Then $L$ is an intermediate field of $\quotient{E}{F}$.
\end{theorem}
\begin{proof}
If $\alpha, \beta \in L$ with $\beta \neq 0$, we need to show that $\alpha \pm
\beta, \alpha\beta, \frac{\alpha}{\beta} \in L$. By definition of $L$, we have
$[F(\alpha)] < \infty$ and $[F(\beta) : F] < \infty$. Consider the field
$F(\alpha, \beta)$. Since the minimal polynomial of $\alpha$ over $F(\beta)$
divides the minimal polynomial of $\alpha $ over $F$ (the minimal polynomial
$\alpha$ over $F$, say $p(x) \in F[x]$, is also a polynomial over $F(\beta)$.
In otherwords, $p(x) \in F(\beta)[x]$ such that $p(\alpha) = 0$), we have
\begin{equation*}
[F(\alpha, \beta):F(\beta)] \leq [F(\alpha):F].
\end{equation*}
Combining this with \cref{theorem1}, we have
\begin{align*}
[F(\alpha,\beta):F] &= [F(\alpha,\beta):F(\beta)][F(\beta):F]\\
&\leq [F(\alpha):F][F(\beta):F]
\end{align*}
Since $\alpha + \beta \in F(\alpha, \beta)$, it follows that
\begin{equation*}
[F(\alpha + \beta):F] \leq [F(\alpha, \beta):F] < \infty,
\end{equation*}
so $a+b\in L$. We can follow a similar line to show $\alpha - \beta,
\alpha\beta, \frac{\alpha}{\beta} \in L$. So $L$ is a field.
\end{proof}
\begin{definition}
Let $\quotient{E}{F}$ be a field extension. The set,
\begin{equation*}
L \coloneqq \{\alpha \in E: [F(\alpha) : F] < \infty\}
\end{equation*}
is called the \underline{algebraic closure of $F$ in $E$}.
\end{definition}
\begin{definition}
A field $F$ is \underline{algebraically closed} if for any algebraic extension
$\quotient{E}{F}$, we have $E = F$.
\end{definition}
\begin{example}
By the Fundamental Theorem of Algebra, $\C$ is algebraically closed.
\end{example}

\subsection{Eisenstein's Criterion}
\begin{definition}
Let $f(x) = a_nx^n + \dots + a_1x + a_0 \in \Z[x]$. We say $f(x)$ is
\underline{primitive} if $a_n > 0$ and $\gcd(a_0, \dots, a_n) = 1$. 
\end{definition}
\begin{lemma*}
Every non-zero polynomial $f(x) \in \Q[x]$ can be written uniquely as a product
$F(x) = cf_0(x)$ where $c \in \Q$ and $f_0(x)$ is a primitive polynomial on
$\Z[x]$. Moreover, $f(x) \in \Z[x]$ if and only if $c \in \Z$. If so, then
$|c|$ is the greatest common divisor of the coefficients of $f(x)$ and the sign
of $c$ is the sign of the leading coefficient of $f(x)$.
\end{lemma*}
\begin{theorem*}[Gauss' Lemma for {$\Z[x]$}] \label{gauss}
Let $f(x) \in \Z[x]$ be non-constant. If $f(x)$ is irreducible in $\Z[x]$, then
it is irreducible in $\Q[x]$. 
\end{theorem*}
\begin{example}
The converse of \cref{gauss} is not true. Consider the polynomial $2x + 8$ is
irreducible in $\Q[x]$, but $2x+8 = 2(x+4)$ is reducible in $\Z[x]$.
\end{example}
\begin{remark}
$f(x) \in \Z[x]$  is irreducible in $\Z[x]$ if and only if either
\begin{enumerate}
\item $f(x)$ is a prime integer
\item $f(x)$ is a primitive polynomial which is irreducible in $\Q[x]$
\end{enumerate}
\end{remark}
\begin{theorem}[Eisenstein's Criterion for {$\Z[x]$}] \label{eisenstien}
Let $f(x) = a_nx^n + \dots + a_1x + a_0 \in \Z[x]$ and let $p$ be a prime
integer. Suppose that $p \nmid a_n$, $p \mid a_i$ for all $0 \leq i \leq (n-1)$ and
$p^2 \nmid a_0$, then $f(x)$ is irreducible in $\Q[x]$. In particular, if
$f(x)$ is primitive, then it is irreducible in $\Z[x]$.
\end{theorem}
\begin{proof}
Consider the map $f: \Z[x] \to \Z_p[x]$ defined by
\begin{equation*}
f(x) \mapsto \bar f(x) = \bar a_nx^n + \dots + \bar a_1x + \bar a_0
\end{equation*}
where $\bar a_i = a_i \Mod{p} \in \Z_p$. Since $p \nmid a_n$ and $p \mid a_i$
for all $0 \leq i (n-1)$, we have $\bar f(x) = \bar a_nx^n$ with $\bar a_n \neq
0$. If $f(x)$ is reducible in $\Q[x]$, then it can be factored in $\Z[x]$ into
polynomials of positive degree, say $f(x) = g(x)h(x)$ with $g(x), h(x) \in
\Z[x]$ and $\deg(g), \deg(h) \geq 1$. It follows that $\bar a_nx^n = \bar
g(x) \bar h(x)$ from which we see that $\bar g(x)$ and $\bar h(x)$ have no constant
terms in $\Z_p[x]$, as $\Z_p[x]$ is a UFD\@. Since the constants of both $g(x)$
and $h(x)$ are divisible by $p$, this implies that the constant of $f(x)$ is
divisible by $p^{2}$, which leads to a contradiction. So, $f(x)$ is irreducible
in $\Q[x]$
\end{proof}
\begin{example}
The polynomial $2x^7 + 3x^4 + 6x^2 + 12$ is irreducible in $\Q[x]$ by applying
Eisenstein's Criterion with $p=3$.
\end{example}
\begin{example}
Consider the \underline{$n^{\text{th}}$ cyclotomic polynomial} defined by
\begin{equation*}
\Phi_n(x) = \sum_{\substack{1 \leq k \leq n\\\gcd(k,n) = 1}}
\bigg(x-e^{2i\pi\frac{k}{n}} \bigg).
\end{equation*}
If $n = p$ where $p$ is a prime number, then $\xi_p = e^{\frac{2i\pi}{p}} =
\cos{\frac{2\pi}{p}} + i\sin{\frac{2\pi}{p}}$ (the $p^{\text{th}}$ root of 1)
is a root of the $p^{\text{th}}$ cyclotomic polynomial. Notice here, since $p$
is co-prime with all $1 \leq k \leq p$, we have
\begin{equation*}
\Phi_p(x) = x^{p-i} + x^{p-2} + \dots + x + 1 = \frac{x^p -1}{x-1}
\end{equation*}
Eisenstein's Criterion does not imply the irreducibility of $\Phi_p(x)$
immediately; however, consider
\begin{equation*}
\Phi_p(x+1) = \frac{(x+1)^p -1}{x} = x^{p-1} + \binom{p}{1}x^{p-2} + \dots +
\binom{p}{p-2}x + \binom{p}{p-1}\in \Z[x]
\end{equation*}
with the Binomial Theorem. Since $p$ is prime, $p \nmid 1$, $p \mid
\binom{p}{i}, \forall i \in \{1,\dots,p-1\}$ and $p^2 \nmid \binom{p}{p-1}$.
Here, Eisenstein's Criterion gives that $\Phi_p(x+1)$ is irreducible in
$\Q[x]$, but if $\Phi_p(x) = g(x)f(x)$, then $\Phi_p(x+1) = g(x+1)h(x+1)$ gives
a factorization for $\Phi_p(x+1)$, so $\Phi_p(x)$ must be irreducible in
$\Q[x]$ as well. Furthermore, since $\Phi_p(x)$ is primitive, $\Phi_p(x)$ is
also irreducible in $\Z[x]$.
\end{example}
\begin{example}
Let $p$ be prime and $\xi_p = e^{\frac{2i\pi}{p}}$. Since it is a root of
$\Phi_p(x)$, which is irreducible, by \cref{theorem4},
\begin{equation*}
[\Q(\xi_p):\Q] = \deg(\Phi_p(x)) = p-1.
\end{equation*}
The field $\Q(\xi_p)$ is called the \underline{$p^{\text{th}}$ cyclotomic
extension on $\Q$}.
\end{example}
\begin{example}
Let $\bar\Q$ be the algebraic closure of $\Q$ in $\C$. Since $\xi_p \in \Q$, we
have
\begin{equation*}
[\bar\Q : \Q] \geq [\Q(\xi_p):\Q] = p-1.
\end{equation*}
Since $p \to \infty$, we have $[\bar\Q:\Q]$ is $\infty$. We have seen in
\cref{theorem6} that if $\quotient{E}{F}$ is finite, then $\quotient{E}{F}$ is
algebraic. However, this example shows that the converse is false. 
\end{example}
\sectionline
Now, let $R$ be any unique factorization domain and let $F$ be it's fraction
field. Then $R[x]$ is a subring of $F[x]$.
\begin{lemma*} [Gauss' Lemma]
Let $R$ be a UFD with the fraction field $F$. Let $f(x) \in R[x]$ be
non-constant. If $f(x)$ is irreducible in $R[x]$, then it is irreducible in
$F[x]$.
\end{lemma*}
\begin{theorem} [Eisenstein's Criterion]
Let $R$ be a UFD with the fraction field $F$. Let $\ell$ be an irreducible
element of $R$. If $f(x) = a_nx^n + \dots + a_1x + a_0 \in R[x]$ with $n \geq
1$, $\ell \nmid a_n$, $\ell \mid a_i$, for all $0 \leq i \neq (n-1)$ and
$\ell^2 \nmid a_0$, then $f(x)$ is irreducible in $F[x]$.
\end{theorem}

\pagebreak
\section{Splitting Fields}
\begin{definition}
Let $\quotient{E}{F}$ be a field extension. We say $f(x) \in R[x]$
\underline{splits over $E$} if $E$ contains all roots of $f(x)$. In other words,
$f(x)$ is a product of linear factors in $E[x]$.
\end{definition}
\begin{definition}
Let $\quotient{\tilde{E}}{F}$, $f(x) \in F[x]$ and $F \subseteq E \subseteq
\tilde E$. If
\begin{enumerate}
\item $f(x)$ splits over $E$
\item there is no proper subfield of $E$ such that $f(x)$ splits over $E$, 
\end{enumerate}
then
we say $E$ is a \underline{splitting field of $f(x) \in F[x]$ in $\tilde E$}.
\end{definition}

\subsection{Existence of Splitting Fields}
\begin{theorem}
Let $p(x) \in F[x]$ be irreducible. The quotient ring
$\quotient{F[x]}{\cyclic{p(x)}}$ is a field containing $F$ and a root of
$p(x)$.
\label{theorem10}
\end{theorem}
\begin{proof}
Since $p(x)$ is irreducible, the ideal $I = \cyclic{p(x)}$ is maximal. Thus,
$E = \quotient{F[x]}{I}$ is a field. Consider the map
\begin{equation*}
\Psi:F\to E, \quad a \mapsto a + I
\end{equation*}
Since $F$ is a field and $\Psi \neq 0$, $\Psi$ is injective. Thus, be
identifying $F$ with $\Psi(F)$, $F$ is a subfield of $E$.
\begin{claim}
Let $\alpha = x + I \in E$. Then $\alpha$ is a root of $p(x)$.
\end{claim}
Notice,
\begin{align*}
p(x) &= a_0 + a_1x + \dots + a_nx^n\\
&= (a_0 + I) + (a_1 + I)x + \dots (a_n + I)x^n\\
&\in E[x].
\end{align*}
Thus, we have
\begin{align*}
p(\alpha) &= (a_0 + I) + (a_1 + I)\alpha + \dots + (a_n + I)\alpha^n\\
&= (a_0 + I) + (a_1 + I)(x+I) + \dots + (a_n + I)(x+I)^n\\
&= (a_0 + a_1x + \dots + a_nx^n) + I \tag{since $(x+I)^i = x^i + I$}\\
&= p(x) + I\\
&= 0 + I\\
&= I
\end{align*}
Thus, $\alpha = x+I \in E$ is a root of $p(x)$.
\end{proof}
\begin{theorem} [Kronecker]
Let $f(x) \in F[x]$. There exists a field $E$ containing $F$ such that
$f(x)$ splits over $E$.
\label{theorem11}
\end{theorem}
\begin{proof}
We proceed with induction on $\deg(f)$. If $\deg(f) = 1$, let $E=F$ and we are
done. Suppose $\deg(f)>1$ and the statement holds for all $g(x)$ with
$\deg(g) > \deg(f)$ ($g(x)$ need not to be in $F[x]$). We write $f(x) =
p(x)h(x)$, where $p(x),h(x) \in F[x]$ and $p(x)$ is irreducible. By
\cref{theorem10}, there exists a field $K$ such that $F \subseteq K$ and $K$
containing a root of $p(x)$, say $\alpha$. Thus, $p(x) = (x-\alpha)q(x)$ and
$f(x) = (x-\alpha)g(x)h(x)$ where $q(x) \in K[x]$. Since $\deg(hq) <
\deg(f)$, by induction, there exists a field $E$ containing $K$ over which
$h(x)q(x)$ splits. It follows that $f(x)$ splits over $E$. 
\end{proof}
\begin{theorem}
Every $f(x) \in F[x]$ has a splitting field, which is a finite extension of $F$.
\label{theorem12}
\end{theorem}
\begin{proof}
For $f(x) \in F[x]$, by \cref{theorem11}, there exists a field extension
$\quotient{E}{F}$ over which $f(x)$ splits, say $\alpha_1, \alpha_2, \dots, a_n$
are roots of $f(x) \in E$. Consider $F(\alpha_1,\dots,\alpha_n)$. The field
contains all the roots of $f(x)$ and $f(x)$ does not split over any proper
subfield of it. Thus, $F(\alpha_1,\dots,\alpha_n)$ is the splitting field of
$f(x)$ in $E$. In addition, since $\alpha_i$ are all algebraic,
$\quotient{F(\alpha_1,\dots,\alpha_n)}{F}$ is finite.
\end{proof}

\subsection{Uniqueness of Splitting Fields}
We have seen from \cref{theorem12} that for a field extension $\quotient{\tilde
E}{F}$, a splitting field of $f(x) \in F[x]$ in $E$ is of the form
$F(\alpha_1,\dots,\alpha_n)$ where $\alpha_i$ are roots of $f(x)$ in $\tilde E$.
Thus, it is unique within $\tilde E$.\\

If we change $\quotient{E}{F}$ to a different field extension, sat
$\quotient{E'}{F}$, what is the relation between the splitting field of $f(x)$
in $E$ and the one in $E'$?
\begin{definition}
Let $\phi:R\to R'$ be a ring homomorphism, and $\Phi: R[x] \to R'[x]$ be the
unique ring homomorphism satisfying $\Phi \big|_R = \emptyset$ and $\Phi(x) =
x$. In this case, we say \underline{$\Phi$ extends $\phi$}. More generally, if
$R \subseteq S$, $R' \subseteq S'$, and $\Phi: S \to S'$ is a ring homomorphism
with $\Phi \big|_R = \emptyset$, we say \underline{$\Phi$ extends $\phi$}.
\end{definition}
\begin{theorem}
Let $\phi:F \to F'$ be an isomorphism of fields and $f(x) \in F[x]$. Let
$\Phi: F[x] \to F'[x]$ be the unique ring homomorphism which extends $\phi$. Let
$f'(x) = \Phi(f(x))$ and $\quotient{E}{F}$ and $\quotient{E'}{F'}$ be splitting
fields of $f(x)$ and $f'(x)$ respectively. Then there exists an isomorphism
$\Psi:E\to E'$.
\label{theorem13}
\end{theorem}
\begin{proof}
We proceed with induction on $[E:F]$. If $[E:F] = 1$, then $f(x)$ is a product
of linear factors in $F[x]$, and so is $f'(x)$ in $F'[x]$. Thus, $E = F$ and
$E' = F'$ so take $\Psi = \phi$ and we are done. Now, suppose $[E:F] < 1$ and
the statement is true for all field extensions $\quotient{\tilde E}{\tilde F}$
with $[\tilde E: \tilde F]<[E:F]$. Let $p(x) \in F[x]$ be an irreducible factor
of $f(x)$ with $\deg(p)>1$ and let $p'(x) = \Phi(f(x))$ (such $p(x)$ exists as
if all irreducible factors of $f(x)$ are of degree 1, then $[E:F] = 1$). Let
$\alpha \in E$ and $\alpha' \in E'$ be roots of $p(x)$ and $p'(x)$ respectively.
From \cref{theorem3}, we have an $F$-isomorphism,
\begin{equation*}
F(\alpha) \cong \quotient{F[x]}{\cyclic{p(x)}}, \quad \alpha \mapsto x +
\cyclic{p(x)}
\end{equation*}
Similarly, there is an $F'$-isomorphism,
\begin{equation*}
F'(\alpha) \cong \quotient{F'[x]}{\cyclic{p'(x)}}, \quad \alpha \mapsto x +
\cyclic{p'(x)}
\end{equation*}
Consider the isomorphism $\Phi:F[x] \to F'[x]$ which extends $\phi$. Since
$p'(x) = \Phi(f(x))$, there exists a field isomorphism,
\begin{equation*}
\tilde\Phi: \quotient{F[x]}{\cyclic{p(x)}} \to \quotient{F'[x]}{\cyclic{p'(x)}},
\quad x + \cyclic{p(x)} \mapsto  x + \cyclic{p'(x)}
\end{equation*}
which extends $\phi$. It follows that there exists a field isomorphism,
\begin{equation*}
\tilde\phi: F(\alpha) \to F'(\alpha), \quad \alpha \mapsto \alpha'
\end{equation*}
which extends $\phi$. Note that since $\deg(p) > 2$, $[E:F[\alpha]] < [E:F]$.
Since $E$ (respectively $E'$) is the splitting field of $f(x) \in
F(\alpha)[x]$ (respectively $f(x) \in F(\alpha)[x]$) over $F(\alpha)$
(respectively $F'(\alpha')$), by induction, there exists $\Psi:E \to E'$ which
extends $\tilde \phi$. Thus, $\Psi$ extends $\phi$.
\end{proof}
\begin{corollary}
Any two splitting fields of $f(x) \in F[x]$ over $F$ are $F$-isomorphic. This,
we say ``the'' splitting field of $f(x)$ over $F$.
\end{corollary}
\begin{proof}
Let $\phi:F \to F$ be the identity map and apply \cref{theorem13}
\end{proof}

\subsection{Degree of Splitting Fields}

\begin{theorem}
If $\quotient{E}{F}$ is the splitting field of $f(x)$, then $[E:F] \mid
\deg(f)\fact$.
\label{theorem14}
\end{theorem}
\begin{proof}
We proceed by induction on $\deg(f)$. If $\deg(f) = 1$, choose $E = F$ and we
have $[E:F] \mid 1$. Suppose $\deg(f) < 1$ and the statement holds for all
$g(x)$ with $\deg(g) < \deg(f)$. We break this down into 2 cases.\\
\underline{\bf Case 1}: If $f(x) \in F[x]$ is irreducible and $\alpha \in E$ is
a root of $f(x)$, by \cref{theorem13},
\begin{equation*}
F(\alpha) \cong \quotient{F[x]}{\cyclic{f(x)}} \quad \text{and} [F(\alpha):F] =
\deg(f) = n.
\end{equation*}
We write $f(x) = (x-\alpha)g(x) \in F(\alpha)[x]$ with $g(x) \in
F(\alpha)[x]$. Since $E$ is the splitting field of $g(x)$ over $F[\alpha]$ and
$\deg(g) = n$, by induction hypothesis, $[E:F(\alpha)] \mid (n-1)\fact$. Since
$[E:F] = [E:F(\alpha)][F(\alpha):F]$, it follows that $[E:F] \mid n\fact$.\\
\underline{\bf Case 2}: If $f(x)$ is not irreducible, write $f(x) = g(x)h(x)$
with $g(x),h(x) \in F[x]$, $\deg(g)=m, \deg(h) = k$, $1 \leq m,k, < n$ and
$m + k = n$. Let $K$ be the splitting field of $g(x)$ over $F$. Since $\deg(g) =
m$, by induction, $[K:F] \mid m\fact$. Since $E$ is the splitting field of
$h(x)$ over $K$ and $\deg(h) = k$, by induction hypothesis, $[E:K] \mid k\fact$.
Thus, $[E:F] \mid m\fact k\fact$, which is a factor of $n\fact$ as
\begin{equation*}
\frac{n\fact}{m \fact k\fact} = \binom{n}{m} \in \Z
\end{equation*}
\end{proof}

\pagebreak
\section{Finite Fields}
\subsection{Prime Fields}
\begin{definition}
The \underline{prime field} of a field $F$ is the intersection of all subfields
of $F$.
\end{definition}
\begin{theorem}
If $F$ is a field, then its prime field is isomorphic to either $\Q$ or $\Z_p
\coloneqq \quotient{\Z}{p\Z}$ for some prime $p$.
\label{theorem16}
\end{theorem}
\begin{proof}
Consider the ring map $\chi:\Z \to F$ defined by
\begin{equation*}
\chi(n) = n\cdot 1 = \underbrace{1 + \dots + 1}_{n\,\text{times}}
\end{equation*}
Let $I = \ker(\chi)$, the kernel of $\chi$. Since $\quotient{\Z}{I} \cong
\im(\chi)$, a subring of $F$, it is an integral domain. Thus, $I$ is a prime
ideal. We break this down to two cases.\\
\underline{Case 1}: If $I = \cyclic{0}$, then $\Z \subseteq F$. Since $F$ is a
field, $\Q = \Frac(\Z) \subseteq F$.\\
\underline{Case 2}: If $I = \cyclic{p}$, then
\begin{equation*}
\Z_p = \quotient{\Z}{p\Z} \cong \im{\chi} \subseteq F
\end{equation*}
\end{proof}
\begin{definition}
Given a field $F$, if its prime field is isomorphic to $\Q$ (respectively
$\Z_p$), we say $F$ has \underline{characteristic 0} (respectively
characteristic $p$) denoted by $\ch(F) = 0$ (respectively $\ch(F) = p$).
\end{definition}
\begin{remark}
Note that if $\ch(F) = p$, for $a,b, \in F$,
\begin{equation*}
(a+b)^p = a^p + b^p.
\end{equation*}
Using this property, the following proposition follows.
\end{remark}
\begin{prop}
Let $F$ be a field with $\ch(F) = p$ and let $n\in\N$. Then, the map $\phi:F \to
F$ given by $u \mapsto u^{p^n}$ is an injective $\Z_p$-homomorphism of fields.
If $F$ is finite, then $\phi$ is a $\Z_p$-isomorphism of $F$.
\end{prop}

\subsection{Formal Derivatives and Repeated Roots}
\begin{definition}
If $F$ is a field, the monomials $\{1, x, x^2, \dots\}$ for a $F$-basis of
$F[x]$. Define the linear operator
\begin{equation*}
D:F[x] \to F[x]
\end{equation*}
by $D(1) = 0$ and $D(x^i) = ix^{i-1}$ for all $i \in \N$. Thus for
\begin{equation*}
f(x) = a_0 + a_1x + \dots + a_nx^n, a_i \in F
\end{equation*}
we have
\begin{equation*}
D(f)(x) = a_1 + 2a_2x + \dots + na_nx^{n-1}.
\end{equation*}
Note that
\begin{enumerate}
\item $D$ is linear: $D(f+g) = D(f) + D(g)$
\item $D$ respects the Leibniz Rule: $D(fg) = \big(D(f)\big)g +
f\big(D(g)\big)$.
\end{enumerate}
We call $D(f) = f'$ the \underline{formal derivative} of $f$.
\end{definition}
\begin{theorem}
Let $F$ be a field and $f(x) \in F[x]$.
\begin{enumerate}
\item If $\ch(F) = 0$, then $f'(x) = 0$ if and only if $f(x) = c$ for some $c
\in F$.
\item If $\ch(F) = p$, then $f'(x) = 0$ if and only if $f(x) = g(x^p)$ for some
$g(x) \in F[x]$
\end{enumerate}
\label{theorem18}
\end{theorem}
\begin{proof}
\begin{enumerate}
\item Backwards is trivial. Suppose we have $f(x) = a_0 + a_1x + \dots +
a_nx^n$, then $f'(x) = a_1 + 2a_2x + \dots + na_nx^{n-1} = 0$. This implies that
$ia_i = 0$ for all $1 \leq i \leq n$, Since $\ch(F) = 0$, $i \neq 0$. Thus,
$a_i = 0$ for all $i \geq 1$. This, $f(x) = a_0 \in F$.
\item 
\underline{Forwards}. For $f(x) = a_0 + a_1x + \dots + a_nx^n$, $f'(x) = a_1 +
2a_2x + \dots + na_nx^{x-1} = 0$ implies that $ia_i = 0$ for all $1 \leq i \leq
n$. Since $\ch(F) = p$, $ia_i = 0$ implies that $ia_i = 0$ implies that
$a_i = 0$ unless $p \mid i$. Thus,
\begin{equation*}
f(x) = a_0 + a_px^p + \dots a_{mp}x^{mp} = g(x^p)
\end{equation*}
where $g(x) = a_0 + a_p + \dots +a_{mp}x^m \in F[x]$.\\
\underline{Backwards}. Write $g(x) = b_0 + b_1x + \dots + b_mx^m \in
F[x]$. Then,
\begin{equation*}
f(x) = g(x^p) = b_0 + b_1x^p + \dots + b_mx^{mp}.
\end{equation*}
Thus, $f'(x) = pb_1x^{p-1} + 2pb_2x^{2p-1} + \dots + mpb_m^{pm-1}$. Since
$ch(F) = p$, we have $f'(x) = 0$.
\end{enumerate}
\end{proof}
\begin{definition}
Let $\quotient{E}{F}$ is a field extension and $f(x) \in F[x]$. We say
$\alpha \in E$ is a \underline{repeated root} of $f(x)$ if $f(x) =
(x-\alpha)^2g(x)$ for some $g(x) \in E[x]$.
\end{definition}
\begin{theorem}
Let $\quotient{E}{F}$ is a field extension and $f(x) \in F[x]$. Then $\alpha$ is
a repeated root of $f(x)$ if and only if $(x -\alpha)$ divides both $f$ and
$f'$. In other words, $(x - \alpha) \mid \gcd(f,f')$.
\label{theorem19}
\end{theorem}
\begin{proof}
\underline{Forwards}. Suppose $f(x) = (x-\alpha)^2g(x)$. Then,
\begin{align*}
f'(x) &= 2(x-\alpha)g(x) + (x-\alpha)^2g'(x)\\
&= (x-a) \big(2g(x) + (x-\alpha)g'(x) \big)
\end{align*}
Thus, $(x-\alpha)$ divides both $f$ and $f'$.\\
\underline{Backwards}. Suppose $(x-\alpha)$ divides both $f$ and $f'$. We write
$f(x) = (x-\alpha)h(x)$ where $h(x) \in E[x]$. Then, $f'(x) = h(x) +
(x-\alpha)h'(x)$. Since $f'(\alpha) = 0$, we have $h(\alpha) = 0$. Thus,
$(x-\alpha)$ is a factor of $h(x)$ and $f(x) = (x-\alpha)^2g(x)$ for some
$g(x) \in E[x]$.
\end{proof}
\begin{corollary}
Let $F$ be a field and $f(x) \in F[x]$. Then $f(x)$ has no repeated roots if and
only if $\gcd(f, f') = 1$.
\label{corollary20}
\end{corollary}
\end{document}
