\documentclass[11pt]{article}
\usepackage{notes}
\usepackage{faktor}

\newcommand{\thiscoursecode}{PMATH 351}
\newcommand{\thiscoursename}{Real Analysis}
\newcommand{\thisprof}{Brian Forrest}
\newcommand{\me}{Kevin Cheng}
\newcommand{\thisterm}{Fall 2018}

\hypersetup
{pdfauthor={\me},
pdftitle={\thiscoursecode \thisterm Lecutre Notes},
pdfsubject={\thiscoursename}
pdflang={English}}

\begin{document}
\begin{titlepage}
    \begin{centering}
        {\scshape\LARGE University of Waterloo \par}
        \globe
        {\huge\bf \thiscoursecode}\\
        {\scshape\Large \thiscoursename}\\
        \vspace{.3cm}
        {\scshape Prof. \thisprof~\textbullet~\thisterm\par}
    \end{centering}
\sectionline
\tableofcontents
\sectionline
\thispagestyle{empty}

\pagebreak
\section{Axiom of Choice, Zorn’s Lemma and Cardinality}
\subsection{Basic Notation}
We will introduce some basic material that will be used throughout the rest of
the course.i We will use the following notation
\begin{itemize}
    \item $\N$ will denote the natural numbers $\{1, 2, 3, \dots\}$
	\item $\Z$ will denote the set of integers $\{\dots, -2, -1, 0, 1, 2,
		\dots\}$
	\item $\Q$ will denote the rational numbers $\{\frac{n}{m} : n \in \Z, m \in
		\N\}$
	\item $\R$ will denote the set of real numbers
\end{itemize}

\subsection{Basic Set Theory}
We will use the notation $A\subset B$ and $A\subseteq B$ interchangeably to mean
that $A$ is a subset of $B$ with the possibility that $A=B$ though when we
explicitly wish to emphasize that $A=B$ is a possibility, we will generally use
$A \subseteq B$. When we wish to express that $A$ is a \underline{proper subset}
of $B$, then we can either specify further that $A\neq B$, or we can use the
notation $A \subsetneq B$.

\end{document}
